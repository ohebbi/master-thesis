
To sustain the digital world's increasing computational demand, alternatives to the classical computer must be explored. Quantum computers are commonly thought of as a futuristic device, but are increasingly manifested today as a possible solution. Unfortunately, there are substantial challenges associated with the modern quantum platforms simultaneously as the selection of quantum platforms are slim. The majority of discoveries of potential quantum platforms have happened by accident, and there is an urgent need for new and better materials that can escalate the effort for a sustainable future.

Conveniently, we are progressively recognizing the fourth science paradigm which constitutes of big words like \textit{Big data} and \textit{Data science}, which all comes together into making it possible to extract knowledge from data. In particular, we have during the recent years seen the rise of computational materials science databases due to successful ab-initio approaches alike \textit{density functional theory}. This catalysator has enabled a new approach for materials discovery; instead of calculating properties based on structures, we are now able to reverse the approach into selecting a key property and finding materials that maximise this goal.

In thiss present work, we performed an exploratory analysis for identifying new potential qubit material hosts candidates using machine learning.

In the process of becoming acquainted with the databases, we have developed tools for simple data extraction and processing for six high-throughput databases, including AFLOWlib, AFLOW-ML, Citrination, Materials Project, OQMD and JARVIS-DFT. We utilized the high-throughput code and tools of Matminer to extend a featurization process done by MODnet. Due to a small amount of similar entries in the databases, we apply the featurization procedure to a subsample of $25.000$ materials in the Material Project.

Thereafter, we developed and implemented three approaches to define good and bad candidates, namely the Ferrenti approach, the augmented Ferrenti approach and the insightful approach. For each of the approaches, we applied the dimensionality reduction technique principal component and trained the machine learning algorithms logistic regression, decision tree, random forest and gradient boost. Due to too general and inaccurate criteria in the approaches, we find the Ferrenti approach and the augmented Ferrenti approach not being able to correctly predict properties that favour materials that can fascilitate any quantum effects. However, the insightful approach provides many promising materials such as ZnGeP$_2$, Ge, GeC, BP, LnP, and many more.




\textbf{Future remarks}
Due to an explosive interest in the field of material informatics.

- future remarks
-- new features Matminer
-- better starting materials (e.g. all with included bandstructure/dos)
