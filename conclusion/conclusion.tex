\chapter{Conclusion}

In this present work, we performed an exploratory analysis for identifying new potential qubit material hosts candidates using machine learning.

In the process of becoming acquainted with the databases, we have developed tools for simple data extraction and processing for six high-throughput databases, including AFLOWlib, AFLOW-ML, Citrination, Materials Project, OQMD and JARVIS-DFT. We utilized the high-throughput code and tools of Matminer to extend a featurization process done by MODnet. Due to a small amount of similar entries in the databases, we apply the featurization procedure to a subsample of $25.000$ materials in the Material Project.

Thereafter, we developed and implemented three approaches to define good and bad candidates, namely the Ferrenti approach, the augmented Ferrenti approach and the insightful approach. For each of the approaches, we applied the dimensionality reduction technique principal component and trained the machine learning algorithms logistic regression, decision tree, random forest and gradient boost. We find the Ferrenti approach and the augmented Ferrenti approach not being able to correctly predict properties that favour materials that can fascilitate any quantum effects, since the machine learning trends reveals unconsistent results of both good and bad candidates. We credit this result to the general and inaccurate criteria for the two approaches, in addition to the absent of any features describing potential quantum effets. However, the insightful approach delivers more consistent candidates and provides promising materials such as ZnGeP$_2$, Ge, GeC, BP, LnP, and many more after a manual screening in the post analysis. 

\chapter{Future remarks}



\textbf{Future remarks}
Due to an explosive interest in the field of material informatics.

- future remarks
-- new features Matminer
-- better starting materials (e.g. all with included bandstructure/dos)
