\chapter*{Conclusion}

In this work, we have performed an exploratory analysis for identifying new potential qubit material host candidates using machine learning. In the process of becoming acquainted with the databases, we have developed tools for simple data extraction and processing for six high-throughput databases, including AFLOWlib, AFLOW-ML, Citrination, Materials Project, OQMD, and JARVIS-DFT. We utilized the high-throughput code and tools of Matminer to extend a featurization process done by MODnet. Due to a small number of similar entries in the databases, we apply the featurization procedure to a subsample of $25.000$ materials in the Materials Project.

Thereafter we developed and implemented three approaches to define suitable and unsuitable candidates, namely the Ferrenti approach, the augmented Ferrenti approach, and the insightful approach. For each of the approaches, we applied the dimensionality reduction technique principal component and trained the machine learning algorithms logistic regression, decision tree, random forest, and gradient boost. We find the Ferrenti approach and the augmented Ferrenti approach not being able to correctly predict properties that favor materials that can facilitate any quantum effects, since the machine learning trends reveal inconsistent results of both suitable and unsuitable candidates. We credit this result to the general criteria which are not based on physical principles for the two approaches, in addition to the absence of any features describing potential quantum effects. However, the insightful approach delivers more consistent candidates and predicts $214$ materials, including ZnGeP$_2$, BC$_2$N, MgSe, CdS, BP, Ge, GeC, InP, and InAs, as promising for QT. Additionally, we find that all models in the three approaches agree on $47$ suitable materials, where $8$ are elemental (unary), $29$ are binary and $10$ are tertiary. We suggest these materials as the most promising candidates for future experimental synthesis of novel qubit materials hosts. %, and many more after a manual screening in the post analysis. We suggest these materials
\clearpage

\section*{Related works}

To the best of our knowledge, we are the first to suggest a trained supervised model for the identification of possible novel materials that can be utilized in quantum technology. However, we find other academic studies involving identifying promising novel material hosts, such as the work of \citeauthor{Ferrenti2020} \cite{Ferrenti2020} (which we have reproduced as the Ferrenti approach). They suggest a data mining approach in where they identify a total of $541$ viable hosts present in the Materials Project. In other studies, \citeauthor{FreyNathanC2020MLDo} \cite{FreyNathanC2020MLDo} develops an approach using machine learning, deep transfer learning and first-principles calculations to assess and predict the stability of point defects in 2D materials. Their method of generating features of the point defects is done with Matminer, similar to this work.

%physics-informed representation of defect structures in terms of easily accessible chemical and structural information

\section*{Future prospects}

The research field of materials informatics is currently blooming, and we find exciting projects around every corner.
%The field of materials informatics are in the very beginning, and we are experiencing the start of a blooming research field. Potential research projects that might
Due to the time restriction regarding producing a thesis, we have  disregarded potential research paths and made compromises during this process. Here, we provide a brief list of potential future prospects that can either result from or complement our work.
%the to-do list is packed with possible future improvements and implementations.
\begin{itemize}
  \item In this work, we applied supervised algorithms to find new candidates for quantum technology. Due to the supervised approach, we were dependent on defining suitable or unsuitable candidates. Another approach is to apply unsupervised learning to the data and investigate if there are potential candidates that are grouped with known suitable candidates.
  \item There exists a myriad of potential featurizers in Matminer, and we have only used a handful of them. Therefore, a potential new work would be to add new features from Matminer and/or other HT-DFT databases to provide a larger feature space. Similarly, one can also choose a smaller set of features to see if one obtains similar results.
  \item Construct a new data set with better initial conditions, e.g. only choose compositions that have a calculated electronic structure and density of state. This will result in a smaller dataset, but with potentially higher data quality.
  \item Apply machine learning algorithms to predict properties that favor novel hosts for quantum technology. Relevant for this work would be to make a model that can predict the spin-orbit coupling of materials. Importantly, this leads to the question; \textit{can also other properties which lead to quantum effects be quantified and consequently be predicted?}
  %\item Considering the continuous update of existing and new features in Materials Project, it would be interesting to implement a continuous update to this project as well.
\end{itemize}

%\textbf{Future remarks}
%Due to an explosive interest in the field of material informatics.

%- future remarks
%-- new features Matminer
%-- better starting materials (e.g. all with included bandstructure/dos)
