\chapter*{Conclusion}

In this work, we have performed an exploratory analysis for identifying new potential qubit material hosts candidates using machine learning. In the process of becoming acquainted with the databases, we have developed tools for simple data extraction and processing for six high-throughput databases, including AFLOWlib, AFLOW-ML, Citrination, Materials Project, OQMD and JARVIS-DFT. We utilized the high-throughput code and tools of Matminer to extend a featurization process done by MODnet. Due to a small amount of similar entries in the databases, we apply the featurization procedure to a subsample of $25.000$ materials in the Material Project.

Thereafter, we developed and implemented three approaches to define suitable and unsuitable candidates, namely the Ferrenti approach, the augmented Ferrenti approach and the insightful approach. For each of the approaches, we applied the dimensionality reduction technique principal component and trained the machine learning algorithms logistic regression, decision tree, random forest and gradient boost. We find the Ferrenti approach and the augmented Ferrenti approach not being able to correctly predict properties that favour materials that can fascilitate any quantum effects, since the machine learning trends reveal unconsistent results of both suitable and unsuitable candidates. We credit this result to the general criteria which are not based on physical principles for the two approaches, in addition to the absence of any features describing potential quantum effects. However, the insightful approach delivers more consistent candidates and predicts $214$ materials, including as ZnGeP$_2$, BC$_2$N, BP, Ge, GeC, InP and InAs, as promising for QT. Additionally, we find that all models in the three approaches agree on $47$ suitable materials, where $8$ are elemental (unary), $29$ are binary and $10$ are tertiary. We suggest these materials as the most promising candidates for future experimental synthesis of novel qubit materials hosts. %, and many more after a manual screening in the post analysis. We suggest these materials
\clearpage
\section*{Future prospects}

Due to the time restriction regarding producing a thesis, the to-do list is packed with possible future improvements and implementations.
\vspace{10pt}

\textbf{TODO:}
\begin{itemize}
  \item Apply unsupervised learning to the data and investigate if there are potential candidates that are grouped together with known suitable candidates.
  \item Add new features from Matminer and other HT-DFT to provide a larger feature space. Additionally, choose a smaller set of features to see if it is possible to describe similar results with fewer features.
  \item Construct a new data set with better initial conditions, e.g. only choose compositions that have a calculated electronic structure and density of state. This will result in a smaller dataset, but with potential higher data quality.
  \item Apply machine learning algorithms to predict missing key properties in the data, e.g. spin orbit coupling. Can also other  quantum properties be quantified and consequently be predicted?
  \item Predict the stability of potential defects in the sites of a given structure.
\end{itemize}

%\textbf{Future remarks}
%Due to an explosive interest in the field of material informatics.

%- future remarks
%-- new features Matminer
%-- better starting materials (e.g. all with included bandstructure/dos)
