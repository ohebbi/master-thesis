\chapter{Extraction and featurization of data}

The initial step for gathering and building features can be visualised through the flochart in figure \ref{fig:flowchart-makedata}. Initially, we start by extracting all entries in the Materials Project that matches a specific query. Thereafter, we apply Matminer's featurization tools to make thousands of features of the data. In a parallel step, entries that are deemed similar to the entries from the initial Materials Project query are extracted from AFLOW, AFLOW-ML, JARVIS-DFT, OQMD and Citrine Informatics. Finally, we combine the steps together as interim data and prepare the data for further analysis.

\setlength{\abovecaptionskip}{5cm}
\begin{figure}[!ht]
\begin{picture}(20,10)

\setlength{\unitlength}{0.17in}
\put(0,-5){\framebox(7,3){\thead{Initial MP query}}}
\put(7,-3.5){\vector(3,0){2}}
\put(9,-3.5){\line(0,3){3}}
\put(9,-3.5){\line(0,-3){3}}
\put(9,-6.5){\line(1,0){17}}

\put(13.5,-5){\framebox(3.5,2){\thead{AFLOW}}}
\put(15.25,-5){\vector(0,-1){1.5}}

\put(18,-5){\framebox(5,2){\thead{AFLOW-ML}}}
\put(20.5,-5){\vector(0,-1){1.5}}

\put(12,-10){\framebox(3,2){\thead{OQMD}}}
\put(13.5,-8){\vector(0,1){1.5}}

\put(16,-10){\framebox(5,2){\thead{JARVIS-DFT}}}
\put(18.5,-8){\vector(0,1){1.5}}

\put(22,-10){\framebox(2,2){\thead{CI}}}
\put(23,-8){\vector(0,1){1.5}}

\put(9,-0.5){\vector(1,0){5}}
\put(14,-1.5){\framebox(8,2){\thead{Matminer featurizer}}}
\put(22,-0.5){\line(1,0){4}}

\put(26,-0.5){\vector(0,-1){3}}
\put(26,-6.5){\vector(0,1){3}}

\put(26,-3.5){\vector(1,0){2}}
\put(28,-4.5){\framebox(6,2){\thead{interim data}}}

\end{picture}
\caption{The data flow of the main project, starting from an initial MP-query, and ending with a featurized dataset with entries from several other databases. The matminer featurizer step is further visualized in-depth in figure \ref{fig:flowchart-featurization}.}
\label{fig:flowchart-makedata}
\end{figure}
\vskip0cm
\setlength{\abovecaptionskip}{0cm}


The initial query has the requirement that all entries has to be derived from an experimental ICSD entry, and is reasoned by that we can identify equivalent entries in other databases. Furthermore, all entries in the Materials Project needs to have a band gap larger than $0.1$eV. Recall that Materials Project applies the functional GGA in estimating the band gap, which is known to severely underestimate the given electronic property. Therefore, we have chosen a low value to not rule out any potential candidates but high enough to leave out all materials that can be considered metallic.

\section{Practical data extraction with Python-examples}

For this section, we will show practical examples of how to extract data that might fulfill the criteria for a material to host a qubit candidate given in the theory part. We will begin with the database of Materials Project, and then search for entries in other databases that match entries from MP. This process is reproducable as a jupyter notebook\footnote{add and insert DOI for JN 01-generateDataset-notebook.ipynb} and the databases in question are the ones refered to in the previous section.

Instead of building multiple HTTP-methods from scratch, we will here take a look at the easiest method at obtaining data from each database. This includes looking into the APIs that supports data-extraction and that are recommended by each respective database.

The range of data in a database can consist of data from a few entries up to an unlimited amount of entries with even further optional parameters, and has limitless use in applications. However, the amount of data in a database is irrelevant if the data is inaccessible. Therefore, we provide a qualitative guide in how to extract information in the easiest way possible. The guide will focus on three main bulletpoints; accessibility, speed of extraction and versatility of API. Additionally, the guide will make the reader aware of the current state of documentation that exists of each database. It should be noted that the methods has as ultimate goal to find as many identical entries to the initial MP query, resulting in complicated queries that might be beyond the scope of the available APIs.

Every data extraction class is based on an abstract parent class, which is listed in code listing \ref{lst:base_class}. The advantages of using a base parent class are many, since it improves the readability during code reviews and reduce the main barrier for understanding the underlying structure of a project, while utilising reusable components. Yet, the main advantage of using a base parent class is the fact that it can effortlessly be extended for further implementations since it provides a code skeleton.

The structure of extraction is centered around using the data extraction tools, and not understanding them. Therefore, we only show how to use them here, while the code is found in the Appendix.

\lstinputlisting[language=Python, caption={Base parent class of all data extraction classes. }, label={lst:base_class}]{methodology&implementations/code-listings/class/base_query.tex}


%  we present an elementary formula which we will use in evaluating if a database is accessible or not. It is defined as
%\begin{align}
%  \text{Accessibility} = \frac{\text{Extraction speed}}{\text{Amount of data}}.
%\end{align}
%A large accessibility term implies an ease in extracting information. This formula does not depend on how accessible a database' user interface is, but a discussion of documentation and user interface will be included in the examples.

\subsection{Materials Project}
\label{ssec:materialsproject}

The most up-to-date version of Materials Project can be extracted using the python package pymatgen, which is integrated with Materials Project REST API. Other retrievel tools that is dependent on pymatgen includes matminer, with the added functionality of returning a pandas dataframe. Copies of Materials Project are added frequently to cloud services such as Citrine Informatics, but the latest added entries to Materials Project cannot be guaranteed in such a query.

Entries in Materials Project are characterized using more than 60 features\footnote{All features can be viewed in the documentation of the project: https://github.com/materialsproject/mapidoc/master/materials}, some features being irrelevant for some materials while fundamental for others. The data is divided into three different branches, where the first can be described as basic properties of materials including over $30$ features, while the second branch describes experimental thermochemical information. The last branch yields information about a particular calculation, in particular information that's relevant for running a DFT script.

To extract information from the database, we will be utilising the module pymatgen. This query supports MongoDB query and projection operators\footnote{https://docs.mongodb.com/manual/reference/operator/query/}, resulting in an almost instant query.

\begin{enumerate}
  \item Register for an account\footnote{https://materialsproject.org}, and generate a secret API-key.
  \item Set the required critera.
  \item Set the wanted properties.
  \item Apply the query.
\end{enumerate}

The code nippet in code listing \ref{lst:MPQuery} resembles steps $2-4$, and is filtered as the inital query. %For this particular query we have set the filter as four items. Firstly, we would like to exclude all spin zero isotopes using the MongoDB operator that matches non of the values specified in the array. Thereafter, we would like to have a compound that is deemed similar to an ICSD entry. All of the resulting entries have be deemed non-magnetic (NM), and lastly, all compounds with polar space groups will be excluded.

\lstinputlisting[language=Python, caption={Practical example of extracting information from Materials Project using pymatgen, resulting in a Pandas DataFrame named entries that contains the properties given after performing a filter on the database. The criteria is given as a JSON, and supports MongoDB operators.}, label={lst:MPQuery}]{methodology&implementations/code-listings/query/MPQuery.tex}

\subsection{Citrine Informatics}

Citrine Informatics is a cloud service, which means that the spectrum of stored information varies broadly. We will access research through open access for institutional and educational purposes. Information in Citrine can be stored using a scheme that is broken down into two sections, with private properties for each entry in addition to common fields that are the same for all entries.% However, the query happens swiftly and is noted as highly accessible.

In this example, we will gather experimental data using the module matminer. The following steps are required to extract information from Citrine Informatics.

\begin{enumerate}
  \item Register for an account\footnote{https://citrination.com}, and generate a secret API-key.
  \item Set the required critera.
  \item Set the wanted properties and common fields.
  \item Apply the query.
\end{enumerate}

The code listed in code listing \ref{lst:CIQuery} gives an easy example to steps $2-4$ with experimental data as filter.

\lstinputlisting[language=Python, caption={Practical example of extracting information from Citrine Informatics using matminer, resulting in a Pandas DataFrame named experimental\_entries that contains the properties given after performing a filter on the database. The criteria is given as a JSON.}, label={lst:CIQuery}]{methodology&implementations/code-listings/query/CIQuery.tex}

\subsection{AFLOW}

The query from AFLOW API \cite{Curtarolo2012} supports lazy formatting, which means that the query is just a search and does not return values but rather an object. This object is then used in the query when asking for values. For every object it is neccessary to request the desired property, consequently making the query process significantly more time-demanding than similar queries using APIs such as pymatgen or matminer for Citrine Informatics. Hence, the accessibility is strictly limited to either searching for single compounds or if the user possess sufficient time.

Matminer's data retrievel tool for AFLOW is currently an ongoing issue \cite{Rosenbrock2017}, thus we present in code listing \ref{lst:AFLOWQuery} a function that extracts information from AFLOW and returns a Pandas DataFrame. In contrast to Materials Project and Citrine Informatics, AFLOW does not require an API-key for a query, which reduces the amount of steps to obtain data. The class searches for an stored AFLOW-data, and initialises a MP-query with the initial criteria if not successful. The resulting query will then be used as input to AFLOW.

\lstinputlisting[language=Python, caption={Practical example of extracting information from AFLOW. The function can extract all information in AFLOW for a given list of compounds, however, it is a slow method and requires consistent internet connection.}, label={lst:AFLOWQuery}]{methodology&implementations/code-listings/query/AFLOWQuery.tex}

\subsection{AFLOW-ML}

In this part, we will be using a machine learning algorithm named AFLOW-ML Property Labeled Material Fragments (PLMF) \cite{Isayev2017} to predict the band gap of structures. This algorithm is compatible with a POSCAR of a compound, which can be generated by the CIF (Crystallographic Information File) that describes a crystal's generic structure. It is possible to download a structure as a poscar by using Materials Project front-end API, but is a cumbersome process to do so individually if the task includes many structures. Extracting the feature of POSCAR is yet to be implemented in the RESful API of pymatgen, thus we demonstrate the versatility of pymatgen with a workaround.

We begin with extracting the desired compounds formula, its material\_id for identification, and their respectful structure in CIF-format from Materials Project. In an iterative process, each CIF-structure is parsed to a pymatgen structure, where pymatgen can read and convert the structure to a POSCAR stored as a Python dictionary. Finally, we can use the POSCAR as input to AFLOW-ML, which will return the predicted band gap of the structure. This iterative process parsing and converting, but is an undemanding process. The function that handles this is presented in code listing \ref{lst:AFLOWMLQuery}. Similar to AFLOW-query, this code listing is dependent on MP-data and will apply for a query if the data is not present.

A significant portion of the process is tied up to obtaining the input-file for AFLOW-ML, and fewer structures will result in an easier process. Nevertheless, we present the following steps in order to receive data from AFLOW-ML.

\begin{enumerate}
  \item Download AFLOWmlAPI\footnote{http://aflow.org/src/aflow-ml/ to the same directory as code listing \ref{lst:AFLOWMLQuery}}.
  \item Getting POSCAR from MP.
  \begin{enumerate}
    \item Apply the query from Materials Project with "CIF", "material\_id" and "full\_formula" as properties.
    \item Insert resulting DataFrame into function defined in code listing \ref{lst:AFLOWMLQuery}.
  \end{enumerate}
    \item Insert POSCAR to AFLOW-ML.
\end{enumerate}
\lstinputlisting[language=Python, caption={Practical example of extracting information from AFLOW-ML. The function will convert a CIF-file (from e.g. Materials Project) to a POSCAR, and will use it as input to AFLOW-ML. In return, one will get the structure's predicted band gap. It should be noted that this requires the AFLOW-ML library in the same directory.}, label={lst:AFLOWMLQuery}]{methodology&implementations/code-listings/query/AFLOWMLQuery.tex}

\subsection{JARVIS-DFT}

The newest version of the JARVIS-DFT dataset can be obtained by requesting an account at the official webpage, but with the drawback that an administrator has to either accept or deny the request. Thus, the accessibility of the database is dependent on if there is an active administrator paying attention to the requests, which is a limitation experienced during this work. Another approach is to download the database through matminer, however with the limitation of not neccessarily having the latest version of the database. A third approach is to download a version of JARVIS-DFT that have been made available for requests the 30.04.2020 at http://figshare.com by Choudhary \textit{et al.} \cite{Choudhary2020}. The author provides tools for extraction, yet not compatible with the latest version of Python (3.8) at the time writing (12.03.2021). Therefore, we provide a tool to extract this data through the use of our base class.

\lstinputlisting[language=Python, caption={Practical example of extracting information from JARVIS-DFT. For this example, we exclude all metals by removing all non-measured band gaps.}, label={lst:JARVISDFTQuery}]{methodology&implementations/code-listings/query/JARVISDFTQuery.tex}

We observe that there is no advanced search filter when loading the database from matminer. The author of matminer regards this as the user's task, and is indeed easily done through the use of the python library Pandas.

\section{Matminer featurization}

Before applying any machine learning algorithm, raw data needs to be transformed into a numerical representation that reflects the relationship between the input and output data. This transformation is known as generating descriptors or features, however, we will in this work adapt the name \textit{featurization}. The open source library of Matminer provides tools to featurize existing features extracted from Materials Project. In this section we will describe how to extract the features from an initial Materials Project query result (see subsection. \ref{ssec:materialsproject}), and the resulting features. It is beyond the scope of this work to go in-depth of each feature since the resulting dataset contains a quantity of several thousand features, but we will here take the liberty to serve a brief overview of the features and refer to each respective citation for more information. The respective table with information regarding $39$ distinct matminer featurizers is situated in the Appendix, table \ref{table:featurizers}.

\setlength{\abovecaptionskip}{10cm}
\begin{figure}[!ht]
\begin{picture}(20,20)

\setlength{\unitlength}{0.17in}
\put(0,-8.5){\vector(3,0){2}}
\put(2,-10.5){\framebox(4,4){\thead{Partition \\ dataset}}}
\put(6,-8.5){\vector(3,0){2}}
\put(8,-10.5){\framebox(5,4){\thead{Get objects \\ from MP}}}
\put(13,-8.5){\line(3,0){1}}
\put(14,-8.5){\line(0,1){6}}
\put(14,-8.5){\line(0,-1){6}}

\put(14,-2.5){\vector(3,0){1}}
\put(15,-3.5){\framebox(9,2){\thead{Featurize composition}}}
\put(24,-2.5){\line(3,0){1}}

\put(14,-5.5){\vector(3,0){1}}
\put(15,-6.5){\framebox(9,2){\thead{Featurize structure}}}
\put(24,-5.5){\line(3,0){1}}

\put(14,-8.5){\vector(3,0){1}}
\put(15,-9.5){\framebox(9,2){\thead{Featurize site}}}
\put(24,-8.5){\line(3,0){1}}

\put(14,-11.5){\vector(3,0){1}}
\put(15,-12.5){\framebox(9,2){\thead{Featurize dos}}}
\put(24,-11.5){\line(3,0){1}}

\put(14,-14.5){\vector(3,0){1}}
\put(15,-15.5){\framebox(9,2){\thead{Featurize band}}}
\put(24,-14.5){\line(3,0){1}}

\put(25,-8.5){\line(0,1){6}}
\put(25,-8.5){\line(0,-1){6}}

\put(25,-8.5){\vector(2,0){1}}
\put(26,-10.5){\framebox(6,4){\thead{Done with \\ all partitons?}}}

\put(29.5, -12.){\makebox{\thead{NO}}}
\put(29,-10.5){\line(0,-1){8}}
\put(29,-18.5){\vector(-1,0){16}}
\put(8,-20.5){\framebox(5,4){\thead{Choose next \\ partition}}}
\put(10.5,-16.5){\vector(0,1){6}}


\put(32, -8.5){\vector(1,0){3}}
\put(32.5, -8.){\makebox{\thead{YES}}}

\end{picture}
\caption{The process of the matminer featurizer step, as seen in figure \ref{fig:flowchart-makedata}. To limit the memory and computational usage, the data is partioned into smaller subsets where the respective objects of featurization are obtained through a query to be used in the following featurization steps. This is iteratively done until all the data has been featurized.}
\label{fig:flowchart-featurization}
\end{figure}
\vskip15cm
\setlength{\abovecaptionskip}{10cm}


To apply matminer's featurization tools, we extend an existing implementation by De Breuck \textit{et al.} \cite{Breuck2021} called the Materials Optimal Descriptor Network (MODNet). The author specifies that MODNet is a supervised machine learning framework for learning material properties based on either composition or crystal structure. To provide the training data for their model, MODNet featurizes (through matminer) structures either from Materials Project or in the form of a structure object made by pymatgen. Their current implementation provides featurization for compositions, structures and sites. However, matminer also provides featurization tools for density of states (DOS) and band structures, therefore we modify MODNet and extend it to fascilitate such featurizations.

One immediate limitation of our extension is that Matminer's tools is dependent on a pymatgen DOS- and bandstructure object. These objects contains information up to $5$MB, and becomes a challenge when dealing with data containing several thousand such objects. This is solved by the required features for matminer's featurization for a subsample of the data, followed by a featurization process of the same subsample. When the feaurization is done, we store the new features and throw away the pymatgen features. This is done iteratively for the entire data set. Thus, a compromise between applying several queries and storing information has been done. The scheme can be visualised as the flow chart seen in figure \ref{fig:flowchart-featurization}.

In the extended version of the featurization process, we eliminate all columns that does not have any entries with physical meaning. This is beneficial for several reasons, such as to reduce memory allocated and to preprocess the data. If there are entries existing with both physical and non-physical for the same column, we replace the non-physical meanings with $-1$ for recognition in a later step. Additionally, we convert columns that are categorical or lacks a numerical representation into a categorical portrayal. Thus, we strive to limit the neccessary steps for further processing of data into a machine learning algorithm.

\section{Preprocessing}



\section{Screen procedure}
kaffe
