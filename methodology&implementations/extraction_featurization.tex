\chapter{Extraction and featurization of data}

The main work of this thesis can visualised through the flowchart in figure (TODO: add flowchart figure). Initially, we start by extracting all entries in the Materials Project that matches a specific query. Thereafter, we apply matminer's featurization tools to make thousands of features of the data. In a parallel step, entries that are deemed similar to the entries from the initial Materials Project query are extracted from AFLOW, AFLOW-ML, JARVIS-DFT and OQMD. Finally, we combine the steps together and prepare the data for further analysis.

The initial query has the requirement that all entries has to be derived from an experimental ICSD entry, and is reasoned by that we can identify equivalent entries in other databases. Furthermore, all entries in the Materials Project needs to have a band gap larger than $0.1$eV. Recall that Materials Project applies the functional GGA in estimating the band gap, which is known to severely underestimate the given electronic property. Therefore, we have a chosen a low value to not rule out any potential candidates but high enough to leave out all materials that can be considered metallic.

\section{Practical data extraction with Python-examples}

For this section, we will show practical examples of how to extract data that fulfill the criteria for a material to host a qubit candidate given in the theory part.

We will begin with the database of Materials Project, and then restrict the query thereafter. This data mining process is reproducable as a jupyter notebook\footnote{add and insert DOI for JN data-mining.ipynb} and the databases in question are the ones refered to in the previous section.

Instead of setting up a multiple HTTP-methods, we will here take a look at the easiest method at obtaining data from each database. This includes looking into the APIs that supports data-extraction and that are recommended by each respective database.

The range of data in a database can consist of data from a few entries up to an unlimited amount of entries with even further optional parameters, and has limitless use in applications. However, the amount of data in a database is irrelevant if the data is inaccessible. Therefore, we provide a qualitative guide in how to extract information in the easiest way possible. The guide will focus on three main bulletpoints; accessibility, speed of extraction and versatility of API. Additionally, the guide will make the reader aware of the current state of documentation that exists of each database.

%  we present an elementary formula which we will use in evaluating if a database is accessible or not. It is defined as
%\begin{align}
%  \text{Accessibility} = \frac{\text{Extraction speed}}{\text{Amount of data}}.
%\end{align}
%A large accessibility term implies an ease in extracting information. This formula does not depend on how accessible a database' user interface is, but a discussion of documentation and user interface will be included in the examples.

\subsection{Materials Project}
\label{ssec:materialsproject}

The most up-to-date version of Materials Project can be extracted using the python package pymatgen, which is integrated with Materials Project REST API. Other retrievel tools that is dependent on pymatgen includes matminer, with the added functionality of returning a pandas dataframe. Copies of Materials Project are added frequently to cloud services such as Citrine Informatics, but the latest added entries to Materials Project cannot be guaranteed in such a query.

Entries in Materials Project are characterized using more than 60 features\footnote{All features can be viewed in the documentation of the project: https://github.com/materialsproject/mapidoc/master/materials}, some features being irrelevant for some materials while fundamental for others. The data is divided into three different branches, where the first can be described as basic properties of materials including over $30$ features, while the second branch describes experimental thermochemical information. The last branch yields information about a particular calculation, in particular information that's relevant for running a DFT script.

To extract information from the database, we will be utilising the module pymatgen. This query supports MongoDB query and projection operators\footnote{https://docs.mongodb.com/manual/reference/operator/query/}, resulting in an almost instant query. Thus, Materials Project is regarded as a highly accessible database.

\begin{enumerate}
  \item Register for an account\footnote{https://materialsproject.org}, and generate a secret API-key.
  \item Install pymatgen in the current environment
  \item Import pymatgen.
  \item Set the required critera.
  \item Set the wanted properties.
  \item Apply the query.
\end{enumerate}

The code nippet in code listing \ref{lst:MPQuery} resembles steps $3-6$, with the additional usage of the library Pandas. For this particular query we have set the filter as four items. Firstly, we would like to exclude all spin zero isotopes using the MongoDB operator that matches non of the values specified in the array. Thereafter, we would like to have a compound that is deemed similar to an ICSD entry. All of the resulting entries have be deemed non-magnetic (NM), and lastly, all compounds with polar space groups will be excluded.

\lstinputlisting[language=Python, caption={Practical example of extracting information from Materials Project using pymatgen, resulting in a Pandas DataFrame named entries that contains the properties given after performing a filter on the database. The criteria is given as a JSON, and supports MongoDB operators.}, label={lst:MPQuery}]{methodology&implementations/code-listings/MPQuery.tex}

\subsection{Citrine Informatics}

Citrine Informatics is a cloud service, which means that the spectrum of stored information varies broadly. We will access research through open access for institutional and educational purposes. Information in Citrine can be stored using a scheme that is broken down into two sections, with private properties for each entry in addition to common fields that are the same for all entries. However, the query happens swiftly and is noted as highly accessible.

In this example, we will gather experimental data using the module matminer. The following steps are required to extract information from Citrine Informatics.

\begin{enumerate}
  \item Register for an account\footnote{https://citrination.com}, and generate a secret API-key.
  \item Install matminer in the current environment.
  \item Import matminer.
  \item Set the required critera.
  \item Set the wanted properties and common fields.
  \item Apply the query.
\end{enumerate}

The code listed in code listing \ref{lst:CIQuery} gives an easy example to steps $2-4$ with experimental data as filter. The resulting query will be returned as a Pandas DataFrame, but it is not neccessary to include the pandas since it is already implemented in the module matminer.

\lstinputlisting[language=Python, caption={Practical example of extracting information from Citrine Informatics using matminer, resulting in a Pandas DataFrame named experimental\_entries that contains the properties given after performing a filter on the database. The criteria is given as a JSON.}, label={lst:CIQuery}]{methodology&implementations/code-listings/CIQuery.tex}

\subsection{AFLOW}

The query from AFLOW API \cite{Curtarolo2012} supports lazy formatting, which means that the query is just a search and does not return values but rather an object. This object is then used in the query when asking for values. For every object it is neccessary to request the desired property, consequently making the query process significantly more time-demanding than similar queries using APIs such as pymatgen or matminer for Citrine Informatics. Hence, the accessibility is strictly limited to either searching for single compounds or if the user possess sufficient time.

Matminer's data retrievel tool for AFLOW is currently an ongoing issue \cite{Rosenbrock2017}, thus we present in code listing \ref{lst:AFLOWQuery} a function that extracts information from AFLOW and returns a Pandas DataFrame. In contrast to Materials Project and Citrine Informatics, AFLOW does not require an API-key for a query, which reduces the amount of steps to obtain data.

\lstinputlisting[language=Python, caption={Practical example of extracting information from AFLOW. The function can extract all information in AFLOW for a given list of compounds, however, it is a slow method and requires consistent internet connection.}, label={lst:AFLOWQuery}]{methodology&implementations/code-listings/AFLOWQuery.tex}

\subsection{AFLOW-ML}

In this part, we will be using a machine learning algorithm named AFLOW-ML Property Labeled Material Fragments (PLMF) \cite{Isayev2017} to predict the band gap of structures. This algorithm is compatible with a POSCAR of a compound, which can be generated by the CIF (Crystallographic Information File) that describes a crystal's generic structure. It is possible to download a structure as a poscar by using Materials Project front-end API, but is a cumbersome process to do so individually if the task includes many structures. Extracting the feature of POSCAR is yet to be implemented in the RESful API of pymatgen, thus we demonstrate the versatility of pymatgen with a workaround.

We begin with extracting the desired compounds formula, its material\_id for identification, and their respectful structure in CIF-format from Materials Project. In an iterative process, each CIF-structure is parsed to a pymatgen structure, where pymatgen can read and convert the structure to a POSCAR stored as a Python dictionary. Finally, we can use the POSCAR as input to AFLOW-ML, which will return the predicted band gap of the structure. This iterative process parsing and converting, but is an undemanding process. The function that handles this is presented in code listing \ref{lst:AFLOWMLQuery}.

A significant portion of the process is tied up to obtaining the input-file for AFLOW-ML, and fewer structures will result in an easier process. Nevertheless, we present the following steps in order to receive data from AFLOW-ML.

\begin{enumerate}
  \item Download AFLOWmlAPI\footnote{http://aflow.org/src/aflow-ml/ to the same directory as code listing \ref{lst:AFLOWMLQuery}}.
  \item Getting POSCAR from MP.
  \begin{enumerate}
    \item Apply the query from Materials Project with "CIF", "material\_id" and "full\_formula" as properties.
    \item Insert resulting DataFrame into function defined in code listing \ref{lst:AFLOWMLQuery}.
  \end{enumerate}
    \item Insert POSCAR to AFLOW-ML.
\end{enumerate}

\lstinputlisting[language=Python, caption={Practical example of extracting information from AFLOW-ML. The function will convert a CIF-file (from e.g. Materials Project) to a POSCAR, and will use it as input to AFLOW-ML. In return, one will get the structure's predicted band gap. It should be noted that this requires the AFLOW-ML library in the same directory.}, label={lst:AFLOWMLQuery}]{methodology&implementations/code-listings/AFLOWMLQuery.tex}

\subsection{JARVIS-DFT}

The newest version of the JARVIS-DFT dataset can be obtained by requesting an account at the official webpage, but with the drawback that an administrator has to either accept or deny the request. Thus, the accessibility of the database is dependent on if there is an active administrator paying attention to the requests. Another approach is to download the database through matminer, however with the limitation of not neccessarily having the latest version of the database. The following steps describes the process of extracting JARVIS-DFT using matminer's convenience loader module, and can be regarded as easily accessible with few lines of code and instantanous download.

\begin{enumerate}
  \item Install matminer in the current environment.
  \item Import matminer.
  \item Load the dataset using code listing \ref{lst:JARVISDFTQuery}.
\end{enumerate}
\lstinputlisting[language=Python, caption={Practical example of extracting information from JARVIS-DFT. For this example, we exclude all metals by removing all non-measured band gaps.}, label={lst:JARVISDFTQuery}]{methodology&implementations/code-listings/JARVISDFTQuery.tex}

We can observe that there is no advanced search filter when loading the database from matminer. The author of matminer regards this as the user's task, which is easily done through the use of the python library Pandas.

\section{Matminer featurization}

Before applying any machine learning algorithm, raw data needs to be transformed into a numerical representation that reflects the relationship between the input and output data. This transformation is known as generating descriptors or features, however, we will in this work adapt the name \textit{featurization}. The open source library of Matminer provides tools to featurize existing features extracted from Materials Project. In this section we will describe how to extract the features from an initial Materials Project query result (see subsection. \ref{ssec:materialsproject}), and the resulting features. It is beyond the scope of this work to go in-depth of each feature since the resulting dataset contains a quantity of several thousand features, but we will here take the liberty to serve a brief overview of the features and refer to each respective citation for more information.

\begin{center}
\begin{longtable}{M{3.0cm} M{3.5cm} M{2.5cm} M{2.5cm}}
\caption[Feasible triples for a highly variable Grid]{Feasible triples for
highly variable Grid, MLMMH.} \label{grid_mlmmh} \\

\hline \multicolumn{1}{c}{\textbf{1a}} & \multicolumn{1}{c}{\textbf{2a}} & \multicolumn{1}{c}{\textbf{3a}} &  \multicolumn{1}{c}{\textbf{4a}}\\ \hline
\endfirsthead

\multicolumn{3}{c}%
{{\bfseries \tablename\ \thetable{} -- continued from previous page}} \\
\hline \multicolumn{1}{|c|}{\textbf{1}} &
\multicolumn{1}{c|}{\textbf{2}} &
\multicolumn{1}{c|}{\textbf{3}} &
\multicolumn{1}{c|}{\textbf{4}} \\ \hline
\endhead

\hline \multicolumn{3}{|r|}{{Continued on next page}} \\ \hline
\endfoot

\hline \hline
\endlastfoot

  \hline
  \hline
  Features & Description & Original reference & Feature reference\\
  \hline
  \textbf{Composition features} & & & \\
  \hline
  AtomicOrbitals & Highest occupied molecular orbital (HOMO) and lowest unoccupied molecular orbital (LUMO). & \cite{Kotochigova1997} & - \\
  AtomicPacking- Efficiency & Packing efficiency. & \cite{Laws2015} & - \\
  BandCenter & Estimation of absolute position of band center using geometric mean of electronegativity. & \cite{Butler1978} & - \\
  ElementFraction & Fraction of each element in a composition. & - & - \\
  ElementProperty & Statistics of various element properties. (From preset("magpie")) & \cite{Ong2013,Ward2016, Deml2016} & - \\
  IonProperty & Maximum and average ionic character. & \cite{Ward2016} & - \\
  Miedema & Formation enthalpies of intermetallic compounds, solid solutions, and amorphous phases using semi-empirical Miedema model. & \cite{Weeber1987} & - \\
  Stoichiometry & $L^p$ norm-based stoichiometric attributes. & \cite{Ward2016} & - \\
  TMetalFraction & Fraction of magnetic transition metals. & \cite{Deml2016} & - \\
  ValenceOrbital & Valence orbital attributes such as the mean number of electrons in each shell. & \cite{Ward2016} & - \\
  YangSolidSolution & Mixing thermochemistry and size mismatch terms of Yang and Zhang (2012). & \cite{Yang2012} & - \\
  \hline
  \textbf{Oxid composition features} &  &  &  \\
  \hline
  Electronegativity- Diff & Statistics on electronegativity difference between anions and cations. & \cite{Deml2016} & - \\
  OxidationStates & Statistics of oxidation states. & \cite{Deml2016} & - \\
  \hline
  \textbf{Structure features} & & & \\
  \hline
  DensityFeatures & Calculate density, volume per atom and packing fraction. & - & - \\
  GlobalSymmetry- Features & Determines spacegroup number, crystal system (1-7) and inversion symmetry. & - & - \\
  RadialDistribution- Function & Calculates the radial distribution function of a crystal system. & - & - \\
  CoulombMatrix & Generate the Coulomb matrix, which is a representation of the nuclear coulombic interaction of the input structure. & \cite{Rupp2012} & - \\
  PartialRadial- Distribution- Function & Compute the partial radial distribution function of a crystal structure & \cite{Schuett2014} & - \\
  SineCoulomb- Matrix & Computes a variant of the coulomb matrix developed for periodic crystals. & \cite{Faber2015} & - \\
  EwaldEnergy & Computes the energy from Coulombic interactions based on charge states of each site. & \cite{Ewald1921} & - \\
  BondFractions & Compute the fraction of each bond in a structure, based on nearest neighbours. & \cite{Hansen2015} & - \\
  \hline
%  \caption{hallo}
%  \label{tab:features}
\end{longtable}
\end{center}


\section{Preprocessing}
kaffe
\section{Screen procedure}
kaffe
