\setlength{\abovecaptionskip}{10cm}
\begin{figure}[!ht]
\begin{picture}(20,20)

\setlength{\unitlength}{0.17in}
\put(0,-8.5){\vector(3,0){2}}
\put(2,-10.5){\framebox(4,4){\thead{Partition \\ dataset}}}
\put(6,-8.5){\vector(3,0){2}}
\put(8,-10.5){\framebox(5,4){\thead{Get objects \\ from MP}}}
\put(13,-8.5){\line(3,0){1}}
\put(14,-8.5){\line(0,1){6}}
\put(14,-8.5){\line(0,-1){6}}

\put(14,-2.5){\vector(3,0){1}}
\put(15,-3.5){\framebox(9,2){\thead{Featurize composition}}}
\put(24,-2.5){\line(3,0){1}}

\put(14,-5.5){\vector(3,0){1}}
\put(15,-6.5){\framebox(9,2){\thead{Featurize structure}}}
\put(24,-5.5){\line(3,0){1}}

\put(14,-8.5){\vector(3,0){1}}
\put(15,-9.5){\framebox(9,2){\thead{Featurize site}}}
\put(24,-8.5){\line(3,0){1}}

\put(14,-11.5){\vector(3,0){1}}
\put(15,-12.5){\framebox(9,2){\thead{Featurize dos}}}
\put(24,-11.5){\line(3,0){1}}

\put(14,-14.5){\vector(3,0){1}}
\put(15,-15.5){\framebox(9,2){\thead{Featurize band}}}
\put(24,-14.5){\line(3,0){1}}

\put(25,-8.5){\line(0,1){6}}
\put(25,-8.5){\line(0,-1){6}}

\put(25,-8.5){\vector(2,0){1}}
\put(26,-10.5){\framebox(6,4){\thead{Done with \\ all partitons?}}}

\put(29.5, -12.){\makebox{\thead{NO}}}
\put(29,-10.5){\line(0,-1){8}}
\put(29,-18.5){\vector(-1,0){16}}
\put(8,-20.5){\framebox(5,4){\thead{Choose next \\ partition}}}
\put(10.5,-16.5){\vector(0,1){6}}


\put(32, -8.5){\vector(1,0){3}}
\put(32.5, -8.){\makebox{\thead{YES}}}

\end{picture}
\caption{The process of the matminer featurizer step, as seen in figure \ref{fig:flowchart-makedata}. To limit the memory and computational usage, the data is partioned into smaller subsets where the respective objects of featurization are obtained through a query to be used in the following featurization steps. This is iteratively done until all the data has been featurized.}
\label{fig:flowchart-featurization}
\end{figure}
\vskip15cm
\setlength{\abovecaptionskip}{10cm}
