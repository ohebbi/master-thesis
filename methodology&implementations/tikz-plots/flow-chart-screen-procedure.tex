\setlength{\abovecaptionskip}{13cm}

\begin{figure}[!ht]
\begin{picture}(20,20)
\setlength{\unitlength}{0.17in}

\put(16.5,-2){\vector(0,-1){3}}
\put(14.25,-2){\framebox(5,2){\thead{interim data}}}
%\put(6,-5){\framebox(7,2){\thead{Data preparation}}}
%\put(13,-4){\vector(2,0){3.5}}
\put(14,-8){\framebox(5.5,3){\thead{Preprocessed \\ data}}}
\put(16.5,-8){\vector(0,-1){4}}
\put(16.5,-10){\line(-1,0){8}}
\put(16.5,-10){\line(1,0){8}}

\put(8.5,-10){\vector(0,-1){2}}
\put(24.5,-10){\vector(0,-1){2}}

\put(6,-15){\framebox(5,3){\thead{Ferrenti \\approach}}}
\put(12.5,-15){\framebox(8,3){\thead{Augmented Ferrenti\\approach}}}
\put(22,-15){\framebox(5,3){\thead{Insightful\\approach}}}

\put(8.5,-15){\vector(0,-1){2}}
\put(24.5,-15){\vector(0,-1){2}}
\put(16.5,-15){\vector(0,-1){2}}

\put(6,-20){\framebox(5,3){\thead{Train and\\predict}}}
\put(14,-20){\framebox(5,3){\thead{Train and\\predict}}}
\put(22,-20){\framebox(5,3){\thead{Train and\\predict}}}

\put(-2,-6.5){\makebox{\thead{\textbf{Data preparation}}}}
\put(-2,-13.5){\makebox{\thead{\textbf{Data mining}}}}
\put(-3,-18.5){\makebox{\thead{\textbf{Supervised learning}}}}

\put(8.5,-20){\line(0,-1){2}}
\put(24.5,-20){\line(0,-1){2}}
\put(16.5,-20){\vector(0,-1){4}}

\put(8.5,-22){\line(1,0){8}}
\put(24.5,-22){\line(-1,0){8}}

\put(14,-27){\framebox(5,3){\thead{Summary}}}

\end{picture}
\caption{A continuation of the flowchart in figure \ref{fig:flowchart-makedata} that visualise the steps of data preparation, the three approaches in the data mining step and the subsequent supervised learning. Finally, a summary will be provided. From a hierarchical perspective, we find the steps leading up to the preprocessed data as the top level, while each of the approaches are found one level down. This top-down approach enables the development and implementation of additional approaches while exploiting the full functionality of all other components in the project.}
\label{fig:flowchart-screening}
\end{figure}
\vskip20cm
%\setlength{\abovecaptionskip}{20cm}
