Recent developments of high-throughput databases has shown encouraging results for novel materials discovery. %Here,
%Recent applications of machine learning for novel materials discovery have shown encouraging results in dealing with the large data
In this thesis,
we perform an exploratory analysis for finding novel material hosts to be used in quantum technology. We have developed data extraction tools for numerous databases, and constructed over $4800$ physics-informed features for a dataset consisting of more than $25000$ materials.
Furthermore, we have developed and implemented three data mining approaches, termed \textit{the Ferrenti approach}, \textit{the augmented Ferrenti approach} and \textit{the insightful approach} for defining three distinct training sets for the supervised machine learning algorithms logistic regression, decision tree, random forest and gradient boost to be trained on.

We find a lack of consistent results for the Ferrenti approach and the augmented Ferrenti approach due to a too broad formulation of the training set, whereas the restrictions set in the insightful approach proved suitable. All models agreed on $214$ predicted candidates, with examples such as ZnGeP$_2$, MgSe, BP, BC$_2$N, BP, Ge, GeC, InP and InAs. All approaches and all models agreed on a subset of $47$ eligible candidates of $8$ elemental, $29$ binary and $10$ tertiary compounds.

% We suggest the $28$ materials as the most promising novel qubit material hosts candidates present in our dataset.
