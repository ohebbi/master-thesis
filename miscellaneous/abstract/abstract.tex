In this thesis, we perform an exploratory analysis for finding novel material hosts to be used in quantum technology. We have developed data extraction tools for numerous databases, and applied data featurization through the tools of Matminer \cite{Ward2018}, resulting in a dataset of more than $25000\times4800$ in dimension. Furthermore, we have developed and implemented three data mining approaches, termed \textit{the Ferrenti approach}, \textit{the augmented Ferrenti approach} and \textit{the insightful approach} for defining three distinct training sets for the supervised machine learning algorithms logistic regression, decision tree, random forest and gradient boost to be trained on.

We find a lack of consistent results for the Ferrenti approach and the augmented Ferrenti approach due to a too broad formulation of the training set, whereas the restrictions set in the insightful approach proved suitable. All models agreed on $85$ predicted candidates, while all approaches and all models agreed on a subset of $28$ eligible candidates of $1$ elemental, $20$ binary and $7$ tertiary compounds. This subset includes candidates such as ZnGeP$_2$, BP, BC$_2$, RuC, Ge, GeC and InP. We suggest the $28$ materials as the most promising novel qubit material hosts candidates present in our dataset.
