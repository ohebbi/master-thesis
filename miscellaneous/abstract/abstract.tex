%Recent developments of high-throughput databases has shown encouraging results for novel materials discovery. %Here,
%Recent applications of machine learning for novel materials discovery have shown encouraging results in dealing with the large data
%In this thesis,
Semiconductor materials provides a compelling platform for quantum technology, and vast amount of materials and their properties can be found in high-throughput databases.
%Semiconductor materials can be used as a platform for quantum technology, and vast amounts of materials and their properties can be found in high-throughput databases.
However, filtering among these materials in order to find novel candidates for quantum technology is a challenge. Therefore, we provide a framework for automatic discovery of promising solid-state material hosts using machine learning methods.
%We perform an exploratory analysis for finding novel material hosts to be used in quantum technology.
We have developed data extraction tools for numerous databases, and constructed over $4800$ physics-informed features for a dataset consisting of more than $25000$ materials.
Furthermore, we have developed and implemented three data mining approaches, termed \textit{the Ferrenti approach}, \textit{the augmented Ferrenti approach} and \textit{the insightful approach} for defining three distinct training sets for the supervised machine learning algorithms logistic regression, decision tree, random forest and gradient boost to be trained on.

We find a lack of consistent results for the Ferrenti approach and the augmented Ferrenti approach due to an overly broad formulation of the training set, whereas the restrictions set in the insightful approach proved suitable. All models agreed on $214$ predicted candidates, with examples such as ZnGeP$_2$, MgSe, BP, BC$_2$N, BP, Ge, GeC, InP and InAs. All approaches and all models agreed on a subset of $47$ eligible candidates of $8$ elemental, $29$ binary and $10$ tertiary compounds.

% We suggest the $28$ materials as the most promising novel qubit material hosts candidates present in our dataset.


%Semiconductor materials represent one of teh most promising candidates for quantum technology, and recent developments of high-throughput databases has shown encouraging results for novel materials discovery. %Here,

%However, the challenges of fidning new promising
%Semiconductor materials can be used as a platform for quantum technology, and vast amounts of materials and their properties can be found in high-throughput databases. However, filtering among these materials in order to find promising candidates for quantum technology is a challenge. Therefore, we provide a framework for automatic discovery of solid-state material hosts using machine learning methods.

%, with qubits and single photon sources being the prominent ones
%Large scale numerical simulations generate a huge amount of properties for materials.
%Currently, numerical simulations of a wide amount of semiconductors are stored in high-throughput databases
%However, very few candidates have experimentally been found/discovered.
%However, deciding which materials to explore experimentally is challenging.

%Additionally, the amount of data makes it cumbersome to filter among potential candidates.
%Therefore, we provide a framework for automatic discovery of solid-state material hosts using machine learning methods.

%But the amount of data makes it cumbersome to filter among potential candidates.
%Recent developments of high-throughput databases has shown encouraging results for novel materials discovery. %Here,


%Creating qubits are at the core of the development of the working quantum computer, and semiconductor materials have been proven to be one of the most promising candidates of qubits. but there is a jungle of mateirals out there, and filtering good candidates is an experimental grind/challenge. Therefore, we present an automatic method using state-of-the-art machine learning to automatically filter out potential candidates.

%Creating qubits

%Semiconductor materials

%derfor utviklet en metode for å sortere automatisk
