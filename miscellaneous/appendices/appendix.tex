\chapter{Density functional theory}

\section{The Born-Oppenheimer approximation}
\label{appendix:Born-Oppenheimer}

We have already have knowledge of the terms in the many-particle Hamiltonian, and we can begin by separating it into electronic and nuclear parts:

\begin{align}
  \hat{H}^{en} = \overbrace{T_e + U_{ee} + U_{en}}^{\hat{H}^{e}} + \overbrace{T_n + U_{nn}}^{\hat{H}^{n}}.
\end{align}
Starting from the Schrödinger equation, we can formulate separate expressions for the electronic and the nuclear Schrödinger equations.

\begin{align}
  \hat{H^{en}} \Psi_\kappa^{en}(\textbf{r},\textbf{R}) &= E_\kappa^{en}\Psi_\kappa^{en}(\textbf{r},\textbf{R}) \quad \lvert \times \int \Psi^*(\textbf{r},\textbf{R}) d\textbf{r} \\
  \int \Psi_\kappa^*(\textbf{r},\textbf{R}) (\hat{H}^e + \hat{H}^n)\Psi_\kappa(\textbf{r},\textbf{R})\Theta_\kappa(\textbf{R})d\textbf{r} &= E_\kappa^{en} \underbrace{\int \Psi_\kappa ^* (\textbf{r},\textbf{R}) \Psi_\kappa (\textbf{r},\textbf{R}) d\textbf{r}}_{1} \Theta_\kappa(\textbf{R}).
\end{align}

Since $\Theta_\kappa(\textbf{R})$ is independent of the the spatial coordinates to electrons, we get $E_{\kappa}$ as the total energy of the electrons in the state $\kappa$.

\begin{align}
     E_\kappa(\textbf{R}) \Theta_k(\textbf{R}) + \int \Psi_k^*(\textbf{r},\textbf{R})H^n\Psi_k(\textbf{r},\textbf{R})\Theta_k(\textbf{R})d\textbf{r} = E_k^{en} \Theta_k(\textbf{R}).
\end{align}

Now, the final integration term can be simplified by using the product rule, which results in
\begin{align}
    \Big( T_n+T_n^{'} + T_n^{''} +U_{nn} + E_\kappa(\textbf{R}) \Big)\Theta_\kappa(\textbf{R}) = E_\kappa^{en}\Theta_\kappa (\textbf{R}).
\end{align}
If we neglect $T_n'$ and $T_n^{''}$ to lower the computational efforts, we obtain the Born-Oppenheimer approximation with the electronic eigenfunction as
\begin{align}
    \left( T_e + U_{ee} + U_{en} \right) \Psi_\kappa (\textbf{r},\textbf{R}) = E_{\kappa}(\textbf{R})\Psi_\kappa(\textbf{r},\textbf{R})
\end{align}
and the nuclear eigenfunction as
\begin{align}
    \left(T_n + U_{nn} + E_\kappa (\textbf{R}) \right) \Theta_\kappa(\textbf{R})= E_{\kappa}^{en}(\textbf{R})\Theta_\kappa(\textbf{r},\textbf{R}).
\end{align}

%These two equations are coupled together through the total energy, which is a potential in the nuclear equation.
How are they coupled, you might ask? The total energy in the electronic equation is a potential in the nuclear equation.

\section{The variational principle}
\label{appendix:variational-principle}
So far, we have tried to make the time-independent Schrödinger equation easier with the use of an \textit{ansatz}, but we do not neccessarily have an adequate guess for the eigenfunctions and the ansatz can only give a rough estimate in most scenarios. Another approach, namely the \textit{variational principle}, states that the energy of any trial wavefunction is always an upper bound to the exact ground state energy by definition $E_0$.
\begin{align}
  E_0 = \bra{\psi_0 } H \ket{\psi_0} \leq \bra{\psi}H\ket{\psi} = E
  \label{eq:variational}
\end{align}
The eigenfunctions of $H$ form a complete set, which means any normalized $\Psi$ can be expressed in terms of the eigenstates
\begin{align}
  \Psi = \sum_n c_n \psi_n, \quad \textnormal{where} \quad H\psi_n = E_n \psi_n
\end{align}
for all $n = 1,2, ...$. The expectation value for the energy can be calculated as
\begin{align*}
  \bra{\Psi}H\ket{\Psi} &= \bra{\sum_{n}c_n \psi_n} H \ket{\sum_{n'} c_{n'}\psi_{n'}} \\
  &= \sum_n \sum_{n'} c_{n}^* c_{n'} \bra{\psi_n}H\ket{\psi_{n'}} \\
  &= \sum_n \sum_{n'} c_{n}^* E_n c_{n'} \bra{\psi_{n}}\ket{\psi_{n'}} \\
\end{align*}
Here we assume that the eigenfunctions have been orthonormalized and we can utilize $\bra{\psi_{m}}\ket{\psi_{n}}=\delta_{mn}$, resulting in
\begin{align*}
  \sum_n c_n^*c_n E_n = \sum_n \lvert c_n \rvert^2 E_n.
\end{align*}
We have already stated that $\Psi$ is normalized, thus $\sum_n \lvert c_n \rvert ^2 = 1 $, and the expectation value conveniently is bound to follow equation \ref{eq:variational}.
The quest to understand the variational principle can be summarized in a sentence - it is possible to tweak the wavefunction parameters to minimize the energy, or summed up in a mathematical phrase,
\begin{align}
  E_0 = \min_{\Psi \rightarrow \Psi_0} \bra{\Psi}H\ket{\Psi}.
\end{align}

\section{The Hohenberg-Kohn theorems}

\subsection{The Hohenberg-Kohn theorem 1}
\label{appendix:theorem1}

\begin{proof}
  Assume that two external potentials $V_{ext}^{(1)}$ and $V_{ext}^{(2)}$, that differ by more than a constant, have the same ground state density $n_0(r)$. The two different potentials correspond to distinct Hamiltonians $\hat{H}_{ext}^{(1)}$ and $\hat{H}_{ext}^{(2)}$, which again give rise to distinct wavefunctions $\Psi_{ext}^{(1)}$ and $\Psi_{ext}^{(2)}$. Utilizing the variational principle, we find that no wavefunction can give an energy that is less than the energy of $\Psi_{ext}^{(1)}$ for $\hat{H}_{ext}^{(1)}$, that is
  \begin{align}
    E^{(1)} = \bra{\Psi^{(1)}}\hat{H}^{(1)}\ket{\Psi^{(1)}} &< \bra{\Psi^{(2)}}\hat{H}^{(1)}\ket{\Psi^{(2)}} \label{eq:E1}
  \end{align}
  and
  \begin{align}
    E^{(2)} = \bra{\Psi^{(2)}}\hat{H}^{(2)}\ket{\Psi^{(2)}} &< \bra{\Psi^{(1)}}\hat{H}^{(2)}\ket{\Psi^{(1)}}.
    \label{eq:E2}
  \end{align}
  Assuming that the ground state is not degenerate, the inequality strictly holds. Since we have identical ground state densities for the two Hamiltonian's, we can rewrite the expectation value for equation \ref{eq:E1} as
  \begin{align*}
    E^{(1)} &= \bra{\Psi^{(1)}}\hat{H}^{(1)}\ket{\Psi^{(1)}} \\
    &= \bra{\Psi^{(1)}}T + U_{ee} + U_{ext}^{(1)}\ket{\Psi^{(1)}} \\
    &= \bra{\Psi^{(1)}} T + U_{ee} \ket{\Psi^{(1)}} + \int \Psi^{*(1)}(\textbf{r})V_{ext}^{(1)}\Psi^{(1)}(\textbf{r})d\textbf{r} \\
    &= \bra{\Psi^{(1)}} T + U_{ee} \ket{\Psi^{(1)}} + \int V_{ext}^{(1)} n(\textbf{r})d\textbf{r} \\
    &< \bra{\Psi^{(2)}}\hat{H}^{(1)}\ket{\Psi^{(2)}} \\
    &= \bra{\Psi^{(2)}} T + U_{ee} + U_{ext}^{(1)} + \overbrace{U_{ext}^{(2)} - U_{ext}^{(2)} }^{0} \ket{\Psi^{(2)}}\\
    &= \bra{\Psi^{(2)}} T + U_{ee} + U_{ext}^{(2)}\ket{\Psi^{(1)}} + \int \left(V_{ext}^{(1)} - V_{ext}^{(2)}\right) n(\textbf{r})d\textbf{r} \\
    &= E^{(2)} + \int \left(V_{ext}^{(1)} - V_{ext}^{(2)}\right) n(\textbf{r})d\textbf{r}.
  \end{align*}
Thus,
\begin{align}
  E^{(1)} = E^{(2)} + \int \left(V_{ext}^{(1)} - V_{ext}^{(2)}\right) n(\textbf{r})d\textbf{r}
\end{align}
A similar procedure can be performed for $E^{(2)}$ in equation \ref{eq:E2}, resulting in

\begin{align}
  E^{(2)} = E^{(1)} + \int \left(V_{ext}^{(2)} - V_{ext}^{(1)}\right) n(\textbf{r})d\textbf{r}.
\end{align}
If we add these two equations together, we get
\begin{align}
  E^{(1)} + E^{(2)} < E^{(2)} + E^{(1)} &+ \int \left( V_{ext}^{(1)} - V_{ext}^{(2)}n(\textbf{r})d\textbf{r} \right) \nonumber \\  &+ \int \left( V_{ext}^{(2)} - V_{ext}^{(1)}n(\textbf{r})d\textbf{r} \right) \nonumber \\
  E^{(1)} + E^{(2)} < E^{(2)} + E^{(1)},
\end{align}
which is a contradiction. Thus, the two external potentials cannot have the same ground-state density, and $V_{ext}(\textbf{\textbf{r}})$ is determined uniquely (except for a constant) by $n(\textbf{\textbf{r}})$.
\end{proof}

\subsection{The Hohenberg-Kohn theorem 2}
\label{appendix:theorem2}

\begin{proof}
  Since the external potential is uniquely determined by the density and since the potential in turn uniquely determines the ground state wavefunction (except in degenerate situations), all the other observables of the system are uniquely determined. Then the energy can be expressed as a functional of the density.
  \begin{align}
    E[n] = \overbrace{T[n] + U_{ee}[n]}^{F[n]} + \overbrace{U_{en}[n]}^{\int V_{en}n(r)dr}
    %\label{eq:densityfunctional}
  \end{align}
  where $F[n]$ is a universal functional because the treatment of the kinetic and internal potential energies are the same for all systems, however, it is most commonly known as the Hohenberg-Kohn functional.

  In the ground state, the energy is defined by the unique ground-state density $n_0(r)$,
  \begin{align}
    E_0 = E[n_0] = \bra{\Psi_0}H\ket{\Psi_0}.
  \end{align}
  From the variational principle, a different density $n(r)$ will give a higher energy
  \begin{align}
    E_0 = E[n_0] = \bra{\Psi_0}H\ket{\Psi_0} < \bra{\Psi}H\ket{\Psi} = E[n]
  \end{align}
  Thus, the total energy is minimized for $n_0$, and so has to be the ground-state energy.
\end{proof}

\section{Self-consistent field methods}
\label{appendix:self-consistent}

So, the remaining question is, how do we solve the Kohn-Sham equation? First, we would need to define the Hartree potential, which can be found if we know the electron density. The electron density can be found from the single-electron wave-functions, however, these can only be found from solving the Kohn-Sham equation. This \textit{circle of life} has to start somewhere, but where? The process can be defined as an iterative method, \textit{a computational scheme}, as visualized in figure \ref{fig:flowchart}.

\clearpage
\setlength{\abovecaptionskip}{22cm}
\begin{figure}[!ht]
\begin{picture}(-20,-20)

\setlength{\unitlength}{0.17in}
\put(12,-4){\framebox(11,4){\thead{Initialize atomic structure,\\ potentials and \\ settings}}}
\put(17,-4){\vector(0,-1){3}}

\put(13,-10){\framebox(9,3){\thead{Initial guess of $n(\textbf{r})$}}}
\put(17,-10){\vector(0,-1){3}}
\put(10,-16){\framebox(15,3){\thead{Use $n(\textbf{r})$ to calculate effective potential \\ $V_{\text{eff}} = V_H + V_{en} + V_{xc}$}}}
\put(17,-16){\vector(0,-1){2}}
\put(11.5,-21){\framebox(12,3){\thead{Solve Kohn-Sham equations \\ and determine $E[n]$}}}
\put(17,-21){\vector(0,-1){2}}
\put(13, -26){\framebox(9, 3){\thead{Calculate new density \\ $n'(\textbf{r})=\sum_j \lvert \psi_j^{KS} \rvert ^2 $}}}
\put(13, -29.5){\line(-1, 0){8}}
\put(9, -29){\makebox{\thead{NO}}}
\put(5, -29.5){\line(0, 1){18}}
\put(5, -11.5){\vector(1, 0){12}}
\put(17,-26){\vector(0,-1){2}}
\put(13, -31){\framebox(9, 3){\thead{Energy self-consistent? \\ $E\left[ n \right] \sim E\left[ n' \right] $ ?}}}
\put(17,-31){\vector(0,-1){3}}
\put(17.5, -32.5){\makebox{\thead{YES}}}
\put(11, -36){\framebox(13, 2){\thead{ Relaxation of the atomic structure?}}}
\put(11, -35){\line(-1,0){4}}
\put(7, -35){\line(0,-1){10.5}}
\put(7, -45.5){\vector(1,0){5}}
\put(8.5, -34.5){\makebox{\thead{NO}}}
\put(17,-36){\vector(0,-1){3}}
\put(17.5, -37.5){\makebox{\thead{YES}}}
\put(13, -41){\framebox(9, 2){\thead{Forces $\sim 0$ ? }}}
\put(17.5, -42.5){\makebox{\thead{YES}}}
\put(24, -39.5){\makebox{\thead{NO}}}
\put(22, -40){\line(1,0){9}}
\put(31, -40){\vector(0,1){12}}
\put(31, -25){\line(0,1){13.5}}

\put(26.5, -28){\framebox(9, 3){\thead{Displacement of atomic\\ positions}}}
\put(31, -11.5){\vector(-1,0){14}}
\put(17,-41){\vector(0,-1){3}}
\put(12, -47){\framebox(11, 3){\thead{Output energies and forces \\ for given configuration}}}
\end{picture}
\caption{A flow chart of the self-consistent field method for DFT.}
\label{fig:flowchart}
\end{figure}
\vskip12cm
\setlength{\abovecaptionskip}{0cm}


\clearpage

\chapter{Featurization}

\section{Table of featurizers}
\begin{center}
\begin{longtable}{M{3.0cm} M{6.0cm} M{2.5cm}}
\caption[]{This thesis' chosen 39 featurizers from matminer. Descriptions are either found from Ref. \cite{Ward2018} or from the project's Github page. For entries lacking references, we refer to \citeauthor{Ward2018} \cite{Ward2018}.}
\label{table:featurizers} \\
\hline \multicolumn{1}{c}{Features} & \multicolumn{1}{c}{Description} & \multicolumn{1}{c}{Reference} \\
\endfirsthead

\multicolumn{3}{c}%
{{\bfseries \tablename\ \thetable{} -- continued from previous page}} \\
\hline \multicolumn{1}{c}{Features} &
\multicolumn{1}{c}{Description} &
\multicolumn{1}{c}{Reference}\\ \hline
\endhead

\hline \multicolumn{3}{r}{{Continued on next page}} \\ \hline
\endfoot

\hline \hline
\endlastfoot

  \hline
  \hline
  \textbf{Composition features} & & \\
  \hline
  AtomicOrbitals & Highest occupied molecular orbital (HOMO) and lowest unoccupied molecular orbital (LUMO). & \cite{Kotochigova1997}  \\ \hline
  AtomicPacking- Efficiency & Packing efficiency. & \cite{Laws2015}  \\ \hline
  BandCenter & Estimation of absolute position of band center using geometric mean of electronegativity. & \cite{Butler1978} \\ \hline
  ElementFraction & Fraction of each element in a composition. & -  \\
  ElementProperty & Statistics of various element properties. & \cite{Ong2013,Ward2016, Deml2016}  \\
  IonProperty & Maximum and average ionic character. & \cite{Ward2016} \\
  Miedema & Formation enthalpies of intermetallic compounds, solid solutions, and amorphous phases using semi-empirical Miedema model. & \cite{Weeber1987} \\
  Stoichiometry & $L^p$ norm-based stoichiometric attributes. & \cite{Ward2016} \\
  TMetalFraction & Fraction of magnetic transition metals. & \cite{Deml2016}  \\
  ValenceOrbital & Valence orbital attributes such as the mean number of electrons in each shell. & \cite{Ward2016}  \\
  YangSolid- Solution & Mixing thermochemistry and size mismatch terms. & \cite{Yang2012} \\
  \hline
  \textbf{Oxid composition features} &  &  \\
  \hline
  Electronegativity- Diff & Statistics on electronegativity difference between anions and cations. & \cite{Deml2016} \\
  OxidationStates & Statistics of oxidation states. & \cite{Deml2016}  \\
  \hline
  \textbf{Structure features} & & \\
  \hline
  DensityFeatures & Calculate density, volume per atom and packing fraction. & - \\
  GlobalSymmetry- Features & Determines spacegroup number, crystal system (1-7) and inversion symmetry. & - \\
  RadialDistribution- Function & Calculates the radial distribution function of a crystal system. & - \\
  CoulombMatrix & Generate the Coulomb matrix, which is a representation of the nuclear coulombic interaction of the input structure. & \cite{Rupp2012}  \\
  PartialRadial- Distribution- Function & Compute the partial radial distribution function of a crystal structure & \cite{Schuett2014}  \\
  SineCoulomb- Matrix & Computes a variant of the coulomb matrix developed for periodic crystals. & \cite{Faber2015}  \\
  EwaldEnergy & Computes the energy from Coulombic interactions based on charge states of each site. & \cite{Ewald1921}  \\
  BondFractions & Compute the fraction of each bond in a structure, based on nearest neighbours. & \cite{Hansen2015}  \\
  Structural- Heterogeneity & Calculates the variance in bond lengths and atomic volumes in a structure. & \cite{Ward2017}  \\
  MaximumPacking- Efficiency & Calculates the maximum packing efficiency of a structure. & \cite{Ward2017} \\
  Chemical- Ordering & Computes how much the ordering of species differs from random in a structure. & \cite{Ward2017}  \\
  XRDPowder- Pattern & 1D array representing normalized powder diffraction of a structure as calculated by pymatgen. & \cite{Ong2013} \\
  \hline
  \textbf{Site features} & & \\
  \hline
  AGNI- Fingerprints & Calculates the product integral of RDF and Gaussian window function & \cite{Botu2014}  \\
  AverageBond- Angle & Determines the average bond angle of a specific site with its nearest neighbors using pymatgens implementation. & \cite{Jong2016}  \\
  AverageBond- Length & Determines the average bond length between one specific site and all its nearest neighbors using pymatgens implementation. & \cite{Jong2016}  \\
  BondOrientational- Paramater & Calculates the averages of spherical harmonics of local neighbors & \cite{Seko2017, Steinhardt1983}  \\
  ChemEnvSite Fingerprint & Calculates the resemblance of given sites to ideal environment using pymatgens ChemEnv package. & \cite{Waroquiers2017, Zimmermann2017}  \\
  Coordination- Number & The number of first nearest neighbors of a site & \cite{Zimmermann2017}  \\
  CrystalNN- Fingerprint & A local order parameter fingerprint for periodic crystals. & -  \\
  GaussianSymm- Func & Calculates the gaussian radial and angular symmetry functions originally suggested for fitting machine learning potentials. & \cite{Behler2011,Khorshidi2016}  \\
  GeneralizedRadial- Distribution- Function & Computes the general radial distribution function for a site & \cite{Seko2017}  \\
  LocalProperty- Difference & Computes the difference in elemental properties between a site and its neighboring sites. & \cite{Ward2017, Jong2016} \\
  OPSite- Fingerprint & Computes the local structure order parameters from a site's neighbor environment. & \cite{Zimmermann2017} \\
  Voronoi- Fingerprint & Calculates the Voronoi tessellation-based features around a target site. & \cite{Peng2011,Wang2019} \\
  \hline
  \textbf{Density of state features} & & \\
  \hline
  DOSFeaturizer & Computes top contributors to the density of states at the valence and conduction band edges. Thus includes chemical specie, orbital character, and orbital location information. & \cite{Dylla2020} \\
  \hline
  \textbf{Band structure features} & & \\
  \hline
  BandFeaturizer & Converts a complex electronic band structure into quantities such as band gap and the norm of k point coordinates at which the conduction band minimum and valence band maximum occur. & - \\
  \hline
%  \caption{hallo}
%  \label{tab:features}
\end{longtable}
\end{center}


\newpage

\section{Erroneous entries}

\begin{table}[!ht]
\centering
\caption{A table of manually identified entries from Materials Project that experience issues concerning Matminer's featurization tools. These were excluded from the dataset.}
\label{tab:error_entries}
\noindent\makebox[\textwidth]{
  \begin{tabular}{M{3.0cm} M{5.0cm} M{3.0cm}}
  \hline
  \hline
  MPID & Full formula & Reference  \\
  \hline
  mp-555563 & PH$_6$C$_2$S$_2$NCl$_2$O$_4$ & \cite{NoneAvailable2020} \\
mp-583476 & Nb$_7$S$_2$I$_{19}$ & \cite{1NoneAvailable2020} \\
mp-600205 & H$_{10}$C$_5$SeS$_2$N$_3$Cl & - \\
mp-600217 & H$_{80}$C$_{40}$Se$_8$S$_{16}$Br$_8$N$_{24}$ & - \\
mp-1195290 & Ga$_3$Si$_5$P$_{10}$H$_{36}$C$_{12}$N$_4$Cl$_{11}$ & - \\
mp-1196358 & P$_4$H$_{120}$Pt$_8$C$_{40}$I$_8$N$_4$Cl$_8$ & - \\
mp-1196439 & Sn$_8$P$_4$H$_{128}$C$_{44}$N$_{12}$Cl$_8$O$_4$ & - \\
mp-1198652 & Te$_4$H$_{72}$C$_{36}$S$_{24}$N$_{12}$Cl$_4$ & - \\
mp-1198926 & Re$_8$H$_{96}$C$_{24}$S$_{24}$N$_{48}$Cl$_{48}$ & - \\
mp-1199490 & Mn$_4$H$_{64}$C$_{16}$S$_{16}$N$_{32}$Cl$_8$ & - \\
mp-1199686 & Mo$_4$P$_{16}$H$_{152}$C$_{52}$N$_{16}$Cl$_{16}$ & - \\
mp-1203403 & C$_{121}$S$_2$Cl$_{20}$ & - \\
mp-1204279 & Si$_{16}$Te$_8$H$_{176}$Pd$_{8}$C$_{64}$Cl$_{16}$ & - \\
mp-1204629 & P$_{16}$H$_{216}$C$_{80}$N$_{32}$Cl$_{8}$ & - \\
  \hline
  \hline
\end{tabular}
}
\end{table}


\clearpage

%\section{Optimalization}
%\label{appendix:Optimalization}

\begin{comment} % Remove this if not neccessary with large tables of scores
\begin{table}[!ht]
\centering
\caption{Table of scores versus principal components used in the model for the insightful approach.}
\label{tab:01-pc-appendix}
\noindent\makebox[\textwidth]{
\begin{tabular}{M{2.0cm} M{3.0cm} M{3.0cm} M{3.0cm} M{3.0cm} M{3.0cm} }
  \hline
  \hline
   Model & PC & Mean test & Mean recall & Mean precision & Mean f1 \\
  \hline
  LOG & 1   &  $0.64(0.010)$   & $ 0.76(0.007)$ \\
      & 2   &  $0.74(0.020)$  & $ 0.82(0.013)$ \\
      & 5   &  $0.78(0.018)$  & $ 0.84(0.013)$ \\
      & 10  &  $0.79(0.019)$ & $ 0.85(0.013)$ \\
      & 50  &  $0.95(0.014)$ & $ 0.96(0.010)$ \\
      & 176 &  $0.98(0.010)$ & $ 0.99(0.008)$ \\
  \hline
  Optimal & 173 & $0.98(0.008)$ & $ 0.99(0.006)$ \\
  \hline
  DT & 1   &  $0.67(0.015)$  & $ 0.78(0.011)$ \\
     & 2   &  $0.74(0.017)$  & $ 0.83(0.010)$ \\
     & 5   &  $0.77(0.024)$  & $ 0.83(0.019)$ \\
     & 10  &  $0.76(0.028)$  & $ 0.82(0.024)$ \\
     & 50  &  $0.77(0.029)$  & $ 0.82(0.021)$ \\
     & 176 &  $0.72(0.030)$  & $ 0.79(0.023)$ \\
  \hline
  Optimal & 37 & $0.79(0.024)$ & $ 0.84(0.020)$ \\
  \hline
  RF & 1   &  $0.68(0.015)$  & $ 0.78(0.010)$ \\
     & 2   &  $0.75(0.019)$  & $ 0.83(0.011)$ \\
     & 5   &  $0.82(0.023)$  & $ 0.87(0.016)$ \\
     & 10  &  $0.84(0.025)$  & $ 0.88(0.017)$ \\
     & 50  &  $0.90(0.020)$  & $ 0.93(0.014)$ \\
     & 176 &  $0.87(0.016)$  & $ 0.91(0.010)$ \\
  \hline
  Optimal & 50 & $0.90(0.020)$  & $ 0.93(0.014)$ \\
  \hline
  GB & 1   &  $0.68(0.015)$  & $ 0.78(0.010)$ \\
     & 2   &  $0.74(0.019)$  & $ 0.83(0.011)$ \\
     & 5   &  $0.83(0.020)$  & $ 0.87(0.015)$ \\
     & 10  &  $0.85(0.020)$  & $ 0.89(0.015)$ \\
     & 50  &  $0.93(0.017)$  & $ 0.95(0.012)$ \\
     & 176 &  $0.93(0.017)$  & $ 0.95(0.012)$ \\
  \hline
  Optimal & 104 & $0.93(0.014)$ & $ 0.95(0.010)$ \\
  \hline
\end{tabular}
}
\end{table}
\end{comment}
