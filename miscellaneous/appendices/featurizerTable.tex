\begin{center}
\begin{longtable}{M{3.5cm} M{6.5cm} M{2.0cm}}
\caption[]{This thesis' chosen 39 featurizers from matminer. Descriptions are either found from Ref. \cite{Ward2018} or from the project's Github page. For entries lacking references, we refer to \citeauthor{Ward2018} \cite{Ward2018}.}
\label{table:featurizers} \\
\hline \multicolumn{1}{c}{Features} & \multicolumn{1}{c}{Description} & \multicolumn{1}{c}{Reference} \\
\endfirsthead

\multicolumn{3}{c}%
{{\bfseries \tablename\ \thetable{} -- continued from previous page}} \\
\hline \multicolumn{1}{c}{Features} &
\multicolumn{1}{c}{Description} &
\multicolumn{1}{c}{Reference}\\ \hline
\endhead

\hline \multicolumn{3}{r}{{Continued on next page}} \\ \hline
\endfoot

\hline \hline
\endlastfoot

  \hline
  \hline
  \textbf{Composition features} & & \\
  \hline
  AtomicOrbitals & Highest occupied molecular orbital (HOMO) and lowest unoccupied molecular orbital (LUMO). & \cite{Kotochigova1997}  \\ \hline
  AtomicPacking- Efficiency & Packing efficiency. & \cite{Laws2015}  \\ \hline
  BandCenter & Estimation of absolute position of band center using geometric mean of electronegativity. & \cite{Butler1978} \\ \hline
  ElementFraction & Fraction of each element in a composition. & -  \\ \hline
  ElementProperty & Statistics of various element properties. & \cite{Ong2013,Ward2016, Deml2016}  \\ \hline
  IonProperty & Maximum and average ionic character. & \cite{Ward2016} \\ \hline
  Miedema & Formation enthalpies of intermetallic compounds, solid solutions, and amorphous phases using semi-empirical Miedema model. & \cite{Weeber1987} \\ \hline
  Stoichiometry & $L^p$ norm-based stoichiometric attributes. & \cite{Ward2016} \\ \hline
  TMetalFraction & Fraction of magnetic transition metals. & \cite{Deml2016}  \\ \hline
  ValenceOrbital & Valence orbital attributes such as the mean number of electrons in each shell. & \cite{Ward2016}  \\ \hline
  YangSolid- Solution & Mixing thermochemistry and size mismatch terms. & \cite{Yang2012} \\
  \hline
  \textbf{Oxid composition features} &  &  \\
  \hline
  Electronegativity- Diff & Statistics on electronegativity difference between anions and cations. & \cite{Deml2016} \\ \hline
  OxidationStates & Statistics of oxidation states. & \cite{Deml2016}  \\ \hline
  \hline
  \textbf{Structure features} & & \\ \hline
  \hline
  DensityFeatures & Calculate density, volume per atom and packing fraction. & - \\ \hline
  GlobalSymmetry- Features & Determines spacegroup number, crystal system (1-7) and inversion symmetry. & - \\ \hline
  RadialDistribution- Function & Calculates the radial distribution function of a crystal system. & - \\ \hline
  CoulombMatrix & Generate the Coulomb matrix, which is a representation of the nuclear coulombic interaction of the input structure. & \cite{Rupp2012}  \\ \hline
  PartialRadial- Distribution- Function & Compute the partial radial distribution function of a crystal structure & \cite{Schuett2014}  \\ \hline
  SineCoulomb- Matrix & Computes a variant of the coulomb matrix developed for periodic crystals. & \cite{Faber2015}  \\ \hline
  EwaldEnergy & Computes the energy from Coulombic interactions based on charge states of each site. & \cite{Ewald1921}  \\ \hline
  BondFractions & Compute the fraction of each bond in a structure, based on nearest neighbours. & \cite{Hansen2015}  \\ \hline
  Structural- Heterogeneity & Calculates the variance in bond lengths and atomic volumes in a structure. & \cite{Ward2017}  \\ \hline
  MaximumPacking- Efficiency & Calculates the maximum packing efficiency of a structure. & \cite{Ward2017} \\ \hline
  Chemical- Ordering & Computes how much the ordering of species differs from random in a structure. & \cite{Ward2017}  \\ \hline
  XRDPowder- Pattern & 1D array representing normalized powder diffraction of a structure as calculated by pymatgen. & \cite{Ong2013} \\
  \hline
  \textbf{Site features} & & \\
  \hline
  AGNI- Fingerprints & Calculates the product integral of RDF and Gaussian window function & \cite{Botu2014}  \\ \hline
  AverageBond- Angle & Determines the average bond angle of a specific site with its nearest neighbors using pymatgens implementation. & \cite{Jong2016}  \\ \hline
  AverageBond- Length & Determines the average bond length between one specific site and all its nearest neighbors using pymatgens implementation. & \cite{Jong2016}  \\ \hline
  BondOrientational- Paramater & Calculates the averages of spherical harmonics of local neighbors & \cite{Seko2017, Steinhardt1983}  \\ \hline
  ChemEnvSite Fingerprint & Calculates the resemblance of given sites to ideal environment using pymatgens ChemEnv package. & \cite{Waroquiers2017, Zimmermann2017}  \\ \hline
  Coordination- Number & The number of first nearest neighbors of a site & \cite{Zimmermann2017}  \\ \hline
  CrystalNN- Fingerprint & A local order parameter fingerprint for periodic crystals. & -  \\ \hline
  GaussianSymm- Func & Calculates the gaussian radial and angular symmetry functions originally suggested for fitting machine learning potentials. & \cite{Behler2011,Khorshidi2016}  \\ \hline
  GeneralizedRadial- Distribution- Function & Computes the general radial distribution function for a site & \cite{Seko2017}  \\ \hline
  LocalProperty- Difference & Computes the difference in elemental properties between a site and its neighboring sites. & \cite{Ward2017, Jong2016} \\ \hline
  OPSite- Fingerprint & Computes the local structure order parameters from a site's neighbor environment. & \cite{Zimmermann2017} \\ \hline
  Voronoi- Fingerprint & Calculates the Voronoi tessellation-based features around a target site. & \cite{Peng2011,Wang2019} \\
  \hline
  \textbf{Density of state features} & & \\
  \hline
  DOSFeaturizer & Computes top contributors to the density of states at the valence and conduction band edges. Thus includes chemical species, orbital character, and orbital location information. & \cite{Dylla2020} \\
  \hline
  \textbf{Band structure features} & & \\
  \hline
  BandFeaturizer & Converts a complex electronic band structure into quantities such as band gap and the norm of k point coordinates at which the conduction band minimum and valence band maximum occur. & - \\
  \hline
%  \caption{hallo}
%  \label{tab:features}
\end{longtable}
\end{center}
