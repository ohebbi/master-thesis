\chapter{Introduction}
je
The year of the covid-19 pandemic, $2020$, was the year when humans emigrated the majority of their lives over to the internet. School classes, business meetings and social events were rescheduled into online lectures, emojis and comments like ``you're muted, Alan''. This emigration was enabled due to a mature silicon-based technology found in our computers and cell phones, which has been developed and improved over decades. These conventional devices are based on transistors, and can be in either state ON (1) or OFF (0), where we have seen an increased performance due to enhancement of clock frequency and reduction of transistor size as predicted by Moore's law \cite{Moore1965, Pavicic2006}. However, transistors are being mass-produced with the feature size at $5$ nm today, but are expected to reach a critical limit of $3$ nm in the following years \cite{Gwennap2020}.

 %silicone-based infrastructure, existing of transistor-based computers and data centers, after

 %The importance of computing power that fue

To sustain the digital world's increasing computational demand, alternatives to the classical computer must be explored. Quantum computers are commonly thought of as a futuristic device, but are increasingly manifested today as a possible solution. The idea of quantum computers is to pass information in the form of a quantum bit, a \textit{qubit}, which can inhabit any superposition of the states one ($1$) and zero ($0$). Unfortunately, there are substantial challenges associated with the modern quantum platforms simultaneously as the selection of quantum platforms is slim. The majority of discoveries of potential quantum platforms has so far happened by serendipity, and there is an urgent need for new and better materials that can escalate the effort for a sustainable future.

Conveniently, we are progressively recognizing the fourth science paradigm which consists of big words like \textit{Big data} and \textit{Data science}, which all comes together into making it possible to extract knowledge from data. In particular, we have during the recent years seen the rise of computational materials science databases \cite{Curtarolo2012, Curtarolo2012a, Calderon2015, Jain2013, Jain2016, Jain2018, Saal2013, Kirklin2015, Choudhary2020, Allen1987} due to successful many-body methods alike \textit{density functional theory} \cite{Kohn1965}. This catalyst has enabled a new approach for novel materials discovery; instead of calculating properties based on composition and structure, we are now able to reverse the approach into selecting a key property and finding materials that maximise this goal.
Fueled by the new paradigm, we find a new field of material science known as \textit{materials informatics} \cite{Rajan2005}.

%we find a new field of materials science develop into what is known as \textit{materials informatics} \cite{Rajan2005}.
In this work, we perform an exploratory analysis in regards to novel materials discovery for quantum technology (QT). We extract information regarding $25212$ possible semiconductors from the Materials Project \cite{Jain2013, Jain2018} and generate $4800$ features to each material using the materials toolkit Matminer \cite{Ward2018} and the high-throughput (HT) method AFLOW-ML \cite{Isayev2017}.
During this process, we develop extraction tools for HT databases, including AFLOW \cite{Curtarolo2012, Curtarolo2012a, Calderon2015}, Materials Project, OQMD \cite{Saal2013, Kirklin2015} and JARVIS-DFT \cite{Choudhary2020}.

Next, we define three training datasets based on the work of \citeauthor{Ferrenti2020} \cite{Ferrenti2020}, namely \textit{the Ferrenti approach}, \textit{the augmented Ferrenti approach} and \textit{the insightful approach}. The first approach is a reproduction of their data mining process, while in the second approach we try to improve this process. In the third approach, we manually identify known suitable candidates, which results in substantially smaller training sets than the two former approaches. Due to the large dimensionality of the data, we utilize the dimensionality reduction technique named principal components analysis (PCA) for identifying correlated descriptors in the data.

To validate how machine learning (ML) models can learn trends and predict materials, we apply the four ML models logistic regression, decision tree, random forest and gradient boost to reproduce the work of \citeauthor{Balachandran2018} \cite{Balachandran2018}. The validation process is set to predict if experimental data of ABO$_3$ solids take the cubic perovskite, perovskite or nonperovskite structure.

Thereafter, we apply the same supervised ML algorithms to train on each of the three approaches, yielding $12$ models in total. Finally, we predict suitable or unsuitable candidates for each of the models based on the remainder of the data set.
 %For each of the approaches, we apply the supervised machine learning

This thesis is centered around the intriguing question; \textit{is it possible to build a model that predicts potential qubit material hosts?} The answer to this question requires intimate knowledge of the interdisciplinary space of quantum technologies, materials informatics and machine learning, which is the sole purpose of Part \rom{1}.
In Part \rom{2}, we describe the process of extraction and construction of data, and the consecutive division of the data into three seperate experiments, or approaches. In Part \rom{3}, the main findings for each approach is presented and discussed. Finally, in Part \rom{4}, we provide a conclusion of the work with possible future prospects.



%Discovery of current deep defects and material hots by serendipitet. (ref. data-mining artikkl)

%This it The introduction.
%Another coffee.

%Trengs det egentlig undertitler her? Tenketenketenk

%\section{Motivation}
%\section{Holy grail}
%\section{Structure of thesis}

% It is specifically found that about 2/3 of the total progress in computation over the past 40 years has been due to materials/process innovations. More speculatively, materials/process innovation contributes at least 20% of the progress in all areas and the relative contribution of materials/process innovation to overall technological progress has grown in the past few decades https://doi.org/10.1002/cplx.20309
