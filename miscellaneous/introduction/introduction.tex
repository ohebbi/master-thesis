\chapter{Introduction}

To sustain the digital world's increasing computational demand, alternatives to the classical computer must be explored. Quantum computers are commonly thought of as a futuristic device, but are increasingly manifested today as a possible solution. Unfortunately, there are substantial challenges associated with the modern quantum platforms simultaneously as the selection of quantum platforms are slim. The majority of discoveries of potential quantum platforms have so far happened by accident, and there is an urgent need for new and better materials that can escalate the effort for a sustainable future.

Conveniently, we are progressively recognizing the fourth science paradigm which constitutes of big words like \textit{Big data} and \textit{Data science}, which all comes together into making it possible to extract knowledge from data. In particular, we have during the recent years seen the rise of computational materials science databases due to successful ab-initio approaches alike \textit{density functional theory}. This catalysator has enabled a new approach for materials discovery; instead of calculating properties based on composition and structure, we are now able to reverse the approach into selecting a key property and finding materials that maximise this goal.



Discovery of current deep defects and material hots by serendipitet. (ref. data-mining artikkl)

This it The introduction.
Another coffee.

Trengs det egentlig undertitler her? Tenketenketenk

%\section{Motivation}
%\section{Holy grail}
%\section{Structure of thesis}

% It is specifically found that about 2/3 of the total progress in computation over the past 40 years has been due to materials/process innovations. More speculatively, materials/process innovation contributes at least 20% of the progress in all areas and the relative contribution of materials/process innovation to overall technological progress has grown in the past few decades https://doi.org/10.1002/cplx.20309
