\chapter{Predicting novel material hosts for quantum technology}

Using the four algorithms, optimized at each of the three approaches, and applying them to the case of predicting materials as suitable material hosts for QT, yields $12$ sets of results. In this chapter we present sets of representative results for each approach. Because of their length, we provide comprehensive tables of the machine learning classifications of the test sets and the training sets in Ref. \cite{Ohebbi2021}.

\section{The Ferrenti approach}

We first consider the machine learning classification of the test set based on the Ferrenti approach. Out of the known suitable candidates defined for the insightful approach, we find many of them in the Ferrenti training set. Carbon in diamond-like structures is present, but we also find two-dimensional carbon in graphite-like structure labelled as suitable. All structures of Si is defined as suitable candidates, together with one entry of SiC. Of other potentially suitable entries, we find ZnS, ZnSe, ZnO and ZnTe present.

\begin{figure}[!ht]
  \centering
  \begin{forest}
    for tree={l sep=1em, s sep=1em, anchor=center, inner sep=0.3em, fill=red!50, circle}
    [$23623$ compounds, node box, alias=bagging, above=1em
    [Logistic regression,node box,alias=a1
      [$11243$]
      [$12380$,fill=green!50,edge label={node[above=1ex,green arrow]{}}
      ]
    ]
    [Decision tree,node box,alias=a1
      [$12308$]
      [$11315$,fill=green!50,edge label={node[above=1ex,green arrow]{}}
      ]
    ]
    [Random forest,node box,alias=a1
      [$9345$]
      [$14278$,fill=green!50,edge label={node[above=1ex,green arrow]{}}
      ]
    ]
    [Gradient boost,node box,alias=a1
      [$11835$]
      [$11788$,fill=green!50,edge label={node[above=1ex,green arrow]{}}
      ]
    ]
    ]
  \end{forest}
\vspace*{-125mm}
\caption{A figure visualizing the number of predictions of potential material candidates for the Ferrenti approach. The green nodes display the number of predicted good candidates, while the bad candidates are marked red.}
\label{fig:01-predictions}
\end{figure}


The number of predicted candidates are labelled in \autoref{fig:01-predictions}. Logistic regression finds a total of $12380$ suitable candidates, while decision tree is the most conservative with $11315$. Random forest has the most optimistic estimate with $14278$, while gradient boost finds $11835$ suitable candidates. The models seems to agree on $6804$ suitable candidates, however, many of the materials are predicted with the probability of similar proportions to a coin-flip. This is exemplified if we were to raise the minimum bar of a prediction to $0.7$, which would make the models only agree on $3000$ suitable candidates. We have included a histogram displaying the distribution of probabilites on the test set in \autoref{fig:histogram-ferrenti}.
In particular, we find that almost all of random forest's predictions are based on a large uncertainty. This behaviour is explained by the nature of random forest, since random forest bases the predictions on an average of predictions in the ensemble of trees. Variance in the underlying trees will bias predictions close to either zero or one \cite{NiculescuMizil2005}. Thus, all trees need to agree for confident prediction.

\begin{figure}[ht!]
    \centering
    \input{../predicting-solid-state-qubit-candidates/reports/figures/histogram/summary-01-ferrenti-approach.tex}
    \vspace*{-130mm}
    \caption{A histogram displaying the distribution of probabilites for all models based on the Ferrenti approach. If the probability is higher (lower) than $0.5$, we label the material as a suitable (unsuitable) candidate.}
    \label{fig:histogram-ferrenti}
\end{figure}

\noindent Of the known suitable materials that were present in the test set, we find that all models admit almost all materials with a chemical formula matching the known candidates.
%The one exception is the decision tree model, which labels C60 Fullerene, cubic Si, cubic GaAs, AlP, GaP, AlaAs and ZnTe as unsuitable candidates.
This can allow materials with unfortunate structures to be labelled as a suitable candidate by all models. Consequently, the models do not recognize the strict band gap restriction which makes it challenging to fascilitate deep defects. This is visualized in the parallel coordinate plot in \autoref{fig:parallel-coordinates-ferrenti}, where the probability for being labelled a suitable candidate for $250$ random entries with band gap less than $5$ eV is displayed.
Ideally, we would expect that the models would have probabilities lower than $0.5$ for all models when the band gap is lower than $0.5$ eV, which would be expected behaviour based on the training set, but this is not the case. We find that many entries with band gap lower than $0.5$ eV, marked as strong red lines in the parallel histogram, are present as both suitable and unsuitable candidates for all models. %Therefore, it seems the algorithm

%Therefore, it seems the algorithms have found a trend in the data set that does not neccessarily favor deep defects.

\begin{figure}[ht!]
    \centering
    \input{../predicting-solid-state-qubit-candidates/reports/figures/parallel_coordinates/summary-01-ferrenti-approach.pgf}
    \vspace*{-130mm}
    \caption{A parallel coordinate plot of $250$ random entries in the test set with MP-calculated band gap less than $5$ eV, where the columns describe the probability for predicting a material as a suitable candidate. A probability over (under) $0.5$ results in a predicted suitable (unsuitable) candidate. Abbreviations used are logistic regression (LOG), decision tree (DT), random forest (RF), gradient boost (GB) and probability (Prob). The figure is based on the Ferrenti approach.}
    \label{fig:parallel-coordinates-ferrenti}
\end{figure}

\begin{comment}
\begin{table}[!ht]
\centering
\caption{Table of the number of predictions made with the optimal model for the insightful approach. }
\label{tab:timing-extraction}
\noindent\makebox[\textwidth]{
\begin{tabular}{M{3.0cm} M{4.0cm} M{4.0cm}}
  \hline
  \hline
   Model & Optimal number PC & Number of predictions \\
  \hline
  Logistic regression & $145$  & $454$ \\
  Decision trees      &  $3$   & $442$ \\
  Random forest       &  $10 $ & $325$ \\
  Gradient boost      &  $7$   & $699$ \\
  \hline
  \hline
\end{tabular}
}
\end{table}
\end{comment}


\section{The augmented Ferrenti approach}
Then we turn towards the perhaps more liberal augmented Ferrenti approach with the result visualized in \autoref{fig:02-predictions}, where we find the most predicted candidates with $14993$, $14407$, $15351$ and $13788$ for logistic regression, decision tree, random forest and gradient boost, respectively. The probability distribution of the predictions is visualized in \autoref{fig:histogram-augmented-ferrenti}. Three of the models, that is gradient boost, decision tree and logistic regression, are very confident in their labelling of suitable candidates and base their predictions on close to $100\%$ probability. Random forest, on the other hand, experience the same variance as in the Ferrenti approach. We observe a peak between $0.75$ and $0.8$, indicating a larger number of positive predictions. Due to the easier restrictions compared to the Ferrenti approach, we find the large amount of $9227$ entries that the four models agree on.

\begin{figure}[!ht]
  \centering
  \begin{forest}
    for tree={l sep=1em, s sep=1em, anchor=center, inner sep=0.3em, fill=red!50, circle}
    [$22550$ compounds, node box, alias=bagging, above=1em
    [Logistic regression,node box,alias=a1
      [$7557$]
      [$14993$,fill=green!50,edge label={node[above=1ex,green arrow]{}}
      ]
    ]
    [Decision tree,node box,alias=a1
      [$8143$]
      [$14407$,fill=green!50,edge label={node[above=1ex,green arrow]{}}
      ]
    ]
    [Random forest,node box,alias=a1
      [$7199$]
      [$15351$,fill=green!50,edge label={node[above=1ex,green arrow]{}}
      ]
    ]
    [Gradient boost,node box,alias=a1
      [$8762$]
      [$13788$,fill=green!50,edge label={node[above=1ex,green arrow]{}}
      ]
    ]
    ]
  \end{forest}
\vspace*{-120mm}
\caption{A figure visualizing the number of predictions of potential material candidates for the augmented Ferrenti approach. The green nodes display the number of predicted good candidates, while the bad candidates are marked red.}
\label{fig:02-predictions}
\end{figure}


\noindent In the training set, we find a single entry of SiC, Si, GaN, ZnS, GaP, AlAs and AlP, carbon in both diamond- and graphite-like structures and AlN in three different structures. Importantly, the training set inludes a larger variety of known suitable candidates compared to the Ferrenti approach due to admitting more elements in the initial restriction. However, since we also included a larger band gap restriction of $1.5$ eV, we find fewer of each known chemical formula present in the training set.

\begin{figure}[ht!]
    \centering
    \input{../predicting-solid-state-qubit-candidates/reports/figures/histogram/summary-02-augmented-ferrenti-approach.tex}
    \vspace*{-130mm}
    \caption{A histogram displaying the distribution of probabilites for all models based on the augmented Ferrenti approach. If the probability is higher (lower) than $0.5$, we label the material as a suitable (unsuitable) candidate.}
    \label{fig:histogram-augmented-ferrenti}
\end{figure}

The summary of the test set reveals that all of the unlabelled known suitable candidates are, in fact, predicted as suitable candidates. Logistic regression predicts a single exception, as it labels almost all structures present of ZnTe as unsuitable candidates. Unfortunately, due to the large number of suitable candidates, it also reveals potentially unqualified predictions. All models confidently predict NaCl as a suitable candidate, which we believe is unlikely due to the electrostatic interactions between Na and Cl. Furthermore, by enforcing a band gap restriction of $1.5$ eV, we find that all models are predicting suitable candidates that exhibit band gaps substantially lower than $0.5$ eV.

 %which is in fact unsuitable due to the phonon-interactions within the lattice that would substantially increase decoherence. Additionally, we find that this approach also predicts materials with band gap lower than $0.5$ eV as suitable candidates.


\section{The insightful approach}

Finally, we turn to the insightful approach, with the results displayed in \autoref{fig:03-predictions}. The four models predicts radically fewer suitable candidates compared to the two latter approaches, where only $842$, $1197$, $543$ and $596$ materials are predicted suitable by logistic regression, decision tree, random forest and gradient boost, respectively. The large majority of the unsuitable candidates are predicted with high probability except for the random forest model due to the ensemble of trees. All models, however, agree on $214$ suitable candidates, whereas $51$ of them have a MP calculated band gap of $0.5$ eV or smaller. 

%Of the predicted suitable candidates, we find the smallest MP calculated band gap as $0.11$ eV, while it is $0.10$ eV for the entire test data. This material is ZnCu$_2$GeSe$_4$,

%The smallest calculated band gap present as predicted suitable candidate is $0.12$ eV, whereas the smallest band gap in the test data is $0.10$ ev. Therefore, we believe

%However, we find also in this approach the presence of suitable candidates with band gap lower than $0.5$ eV. All the models agree on only $214$ suitable candidates, which reduces to $163$ by imposing the bandgap restriction.

\begin{figure}[!ht]
  \centering
  \begin{forest}
    for tree={l sep=1em, s sep=1em, anchor=center, inner sep=0.3em, fill=red!50, circle}
    [$24544$ compounds, node box, alias=bagging, above=1em
    [Logistic regression,node box,alias=a1
      [$23702$]
      [$842$,fill=green!50,edge label={node[above=1ex,green arrow]{}}
      ]
    ]
    [Decision tree,node box,alias=a1
      [$23347$]
      [$1197$,fill=green!50,edge label={node[above=1ex,green arrow]{}}
      ]
    ]
    [Random forest,node box,alias=a1
      [$24001$]
      [$543$,fill=green!50,edge label={node[above=1ex,green arrow]{}}
      ]
    ]
    [Gradient boost,node box,alias=a1
      [$23948$]
      [$596$,fill=green!50,edge label={node[above=1ex,green arrow]{}}
      ]
    ]
    ]
  \end{forest}
\vspace*{-125mm}
\caption{A figure visualizing the number of predictions of potential material candidates for the insightful approach. The green nodes display the number of predicted good candidates, while the bad candidates are marked red.}
\label{fig:03-predictions}
\end{figure}



\noindent Initially, we begin looking at all materials that are predicted suitable with $85\%$ or larger probability for all models, which are BN, BC$_2$N, CdSe, InAs, CuI and ZnCd$_3$Se$_4$. BN (mp-$1639$) is already present in the training data as a suitable candidate, and therefore we believe the models recognized this as a suitable candidate with high probability. Furthermore, two compositions of CdSe (mp-$2691$ and mp-$1070$) have been predicted as suitable as a consequence of the presence of CdS as suitable candidate in the training set. The element S resides in the same group as Se, and the data shows that the two compounds of CdSe exhibit an MP calculated band gap as $0.5$ ev and $0.55$ eV, respectively.

We also find two compositions with the same chemical formula, the orthorombic coordinated (mp-$629458$) with BC$_2$N$_2$ tetrahedras and the chalcopyrite-like structured BC$_2$N (mp-$1008523$) with BC$_4$ tetrahedras. The first structure is in a polar space group while the latter is not. The band gaps are in MP calculated as $1.85$ eV and $1.65$ eV, respectively. BC$_2$N is known as heterodiamond and is a super hard hybrid of diamond and BN. Additionally, we find both the predicted BN and BC$_2$N next to the cluster of C in \autoref{fig:3d-iso}.
Both structures have, as expected, strong covalent character and have been studied for application as nanostructures \cite{Gao2017}, hydrogen storage \cite{Cai2017} and superhard materials \cite{Li2017, Jiang2020} in ab-initio calculations. Of similar compounds, it has been predicted that the diamond-like structure of BC$_3$N can be a prominent spin qubit material host \cite{WangDuo2020Sqbo}. By creating a boron (B) vacancy, it will immediately lead to an NV center with similar properties as found in the NV center in diamond. If this is also possible for BC$_2$N, remains to be seen. We note that BC$_3$N is not present in MP, and therefore not present in our dataset.

% The structures are still in early development, but might show promising host qualities for use in quantum technology.

InAs (mp-$20305$), CuI (mp-$22895$ and mp-$569346$) and ZnCd$_3$Se$_4$ (mp-$1078597$) are close together at the cluster of ZnS in the three dimensional representation in \autoref{fig:3d-iso}, and have band gaps of $0.30$, $1.18$ and $1.73$ eV, respectively. Single self-assembled InAs quantum dots have already been demonstrated \cite{Liu2018}, and therefore is an exciting possible material to use in quantum technology. To the best of our knowledge, ZnCd$_3$Se$_4$ has yet to be synthesized and contains the toxic element Cd, which could prove challenging for synthesis. CuI, however, has recently been synthesized and has been shown to exhibit remarkable optoelectronic properties \cite{Ahn2020}. Interestingly, the material exhibit a large ionic character, and we find it closer towards other oxides in \autoref{fig:3d-iso}.

By lowering the percentage to $75\%$ or larger probability for all models results in $69$ materials, as visualized in \autoref{tab:03-probability-candidates}, where the predicted suitable candidates involves ternary compound of the formula ABC$_2$. The elements Ga, Cd or Zn takes the A-site, Cu, Sn, Ag or Ge takes the B-site, while S, Te, P and As takes the C atom. Most of the formed compounds involves toxic compounds, with one exception. This exception is ZnGeP$_2$ (mp-$4524$), which is a tetrahedrally coordinated material, chalcopyrite-like structure, with reported MP calculated indirect band gap of $1.2$ eV \cite{Zhang2015} and experimentally reported as $1.99$ eV \cite{Xing1989}.
It crystallizes in a non-polar space group, posesses no magnetic moment, has strong covalent bonds and has been reported as an excellent mid-IR transparent crystal material which is suitable for nonlinear optical applications \cite{Zhang2015}. Importantly, it is possible to integrate sources of photon quantum states based on nonlinear optics \cite{Caspani2017}. An eligible candidate indeed, but it remains unknown if the candidate can provide isolation and shelter to experimentally fascilitate a deep defect with quantum effects.


%, which are ZnGeP$_2$, He, BC$_2$N, N$_2$ and RuC. The
%entries have a sufficiently large band gap and is associated with a low spin orbit coupling due to the small size, but we see that the list includes noble gases. The noble gases are described in the data with no ionic character, no electronegativity, low covalent radius, large band gaps and simple structures. Furthermore, they are missing entries on most of the descriptors and we do not have a feature describing any physics of noble gases. We therefore believe the noble gases can be regarded as outliers, and are therefore not offered additional consideration.


%Lastly, we find RuC (mp-$1009792$) in the rock-salt cubic structure as a predicted candidate, and consists of corner-sharing RuC$_4$ tetrahedras. It is a relatively new and unstudied composition, which is found unstable in terms of energy above hull per atom in MP, and have a calculate indirect band gap of $0.72$ eV \cite{RuC2020}.

%Random forest, due to the average of trees, experience a smaller probability than the other models. If we leave the model out, we gain a list of $65$ predicted suitable candidates for the three other models with probability of at least $80\%$.

The work of \citeauthor{Ferrenti2020} \cite{Ferrenti2020} suggests a list of 541 viable hosts, where we find only a single material present in our list of $66$ candidates in \autoref{tab:03-probability-candidates}. This material is the nontoxic MgSe (mp-$10760$), which crystallize in the rock-salt structure.
It has a MP calculated band gap of $1.98$ eV, and an experimental band gap of $5.6$ eV \cite{SaumGeorge1959}. It consists of spin-zero isotopes in accordance to the criteria set by the authors. We believe this criteria favors spin centers in qubits, where MgSe could be a prominent candidate.

Of the $66$ materials mentioned above, we emphasize the presence of Ge, GeC, BP and InP. Ge in cubic structure (mp-$1198022$) share many similar properties with Si and C as well as sharing periodic column number. In fact, the first transistors was made in germanium to its appealing eletrical properties, but silicon took over as the material of choice for microelectronics due to the outstanding quality of silicone dioxide, which allowed the fabrication and integration of increasingly smaller transistors \cite{Scappucci2020, Pillarisetty2011}. Ge has the highest hole mobility of semiconductors at room temperature, and is therefore considered a key material for the process of extending the chip performance in classical computers beyond the limits imposed by miniaturization \cite{Scappucci2020}.

GeC (mp-$1002164$) has a cubic structure and consists of corner-sharing GeC$_4$ tetrahedras. It is non-magnetic, has a MP reported band gap of $1.849$ eV and is highly covalent. The energy above hull per atom is $0.44$ eV, thus reported unstable. Interestingly, SiC is found as a suitable host material, and we encourage further research of GeC due to its comparable properties.

BP (mp-$1479$, mp-$1008559$) is present in the predictions as cubic and hexagonal structure where both consists of corner-sharing BP$_4$ tetrahedras. The indirect band gaps are calculated in MP as $1.46$ and $1.1$ eV, respectively. They are both nonmagnetic, and share many similar properties as the entries mentioned above.

Lastly, we will mention the prediction of InP (mp-$966800$) as a suitable candidate. The compound inhabit a hexagonal structure with corner sharing InP$_4$ tetrahedras. It has a MP calculated direct band gap of $0.51$ eV, and is considered as one of the most promising candidates of Cd- or Pb- based QDs in the application of display and lighting \cite{Zhang2020a, Won2019}.

By further reducing the probability of predictions down to $50\%$ for all models, we eventually find noble gases and other compounds existing as gas in standard conditions. These gases are described in the data with no ionic character, no electronegativity, low covalent radius, large band gaps and simple structures. Furthermore, there are associated errors with the features for these compounds in Materials Project (which has been adressed in the Materials Project version of V2021.05.13\footnote{https://matsci.org/t/materials-project-database-release-log/1609/18 (Visited on 14.05.2021)}). We believe they can be considered as outliers, and they are therefore not added additional consideration.

%Due to that the models consider them as coin-flip and the large amount of missing features, we believe we can regard them as outliers, and they are therefore not added additional consideration.

\begin{figure}[ht!]
    \centering
    \includegraphics[trim={2.8cm 0cm 0cm 0cm},clip, scale=1]{../predicting-solid-state-qubit-candidates/reports/figures/pca-3d-plots/3d-test-iso.pdf}
    \vspace*{-130mm}
    \caption{A three dimensional scatter plots visualizing the testset $24540$ datapoints, and the isosurface of the decision tree decision boundary. Limegreen indicates suitable candidates, while tomato corresponds to unsuitable candidates. The isosurface represents the probability of a prediction, where all values larger than $0.5$ results in a suitable prediction.}
    \label{fig:3d-3}
\end{figure}

In \autoref{fig:3d-3}, we have visualized the three dimensional scatter plot of the decision tree predicted candidates together with its decision boundary. By visualization, we find that ZnGeP$_2$ and Ge are close by the cluster of ZnS, as described in the previous chapter. Otherwise, the materials are relatively spread out and not belonging to any cluster in three dimensions.
% BC$_2$N is, not surprisingly, close to the C cluster.

Additionally, we find that all models agree on several oxides being potential candidates. However, in the visualization, we find that almost all oxides are inbetween the decision boundary defining suitable and unsuitable candidates. Due to the labelling of the suitable candidate ZnO, we believe that the boundary were shifted sufficiently to admit several oxides as suitable candidates.


\section{Comparison of the approaches}

Out of the three approaches, we find that the augmented approach is the least restricted approach and admits the most entries. The Ferrenti approach also admits a large amount of entries, and is considered to not be very different from the Augmented Ferrenti approach. The models in the two approaches are unable to reproduce the criteria that the approaches are based on, such as band gap restriction or polar space group. Of course, the materials that the two initial approaches label as suitable candidates are challenging to go through due to their extensive lengths, whereas the insightful approach predicts fewer suitable candidates and we are able to manually verify many of the compounds.

However, we note that we found predicted suitable candidates with band gap lower than $0.5$ eV for the insightful approach as well, but to a smaller extent. Thus, all three approaches predicted suitable candidates with band gap lower than the lowest in the training sets. We believe there are three reasons that leads to this result. Firstly, we found that the GGA functional Materials Project applies is underestimating the band gap with $30-60\%$, and therefore there is a chance that the models consider the band gap as noise and not useful information. Secondly, we did not find the presence of the band gap of major importance in the principal components, consequently there might be a chance that the band gap is correlated with other features. Thirdly, there are reasons to believe that the models find other patterns that represent a better distinction between suitable and unsuitable candidates in the training sets, resulting in the band gap being redundant.

Of the $214$ suitable candidates predicted by all models in the insightful approach, we find $119$ of them also predicted as suitable by all models in the augmented Ferrenti approach. Similarly, $78$ of them are also predicted as suitable by all the models in the Ferrenti approach. All approaches and their corresponding models agree on $47$ potential candidate, where eight are elementary (unary), $29$ binary and $10$ tertiary.

\begin{center}
\begin{longtable}{M{3.0cm} M{3.0cm} M{3.5cm}}
\caption[]{A table displaying the $66$ predicted candidates that all models in the insightful approach agreed on with more than $75\%$ probability. All band gaps (BG) are found from Materials Project, and materials can appear several times on the list due to different structures. The list involves nine elementary (unary), $45$ binary and $14$ ternary compounds.}
\label{tab:03-probability-candidates} \\
\hline
\hline
Compound formula & MP ID & MP Calculated BG [eV] \\
\hline

\endfirsthead

\multicolumn{3}{c}%
{{\bfseries \tablename\ \thetable{} -- continued from previous page}} \\
\hline
\hline
Compound formula & MP ID & MP Calculated BG [eV]
\\ \hline
\endhead

\hline \multicolumn{3}{r}{{Continued on next page}} \\ \hline
\endfoot

\hline \hline
\endlastfoot
  Ge & mp-137 & 0.87\\
  CdTe & mp-406 & 1.22\\
  HgSe & mp-820 & 0.12\\
  GeTe & mp-938 & 0.82\\
  MgTe & mp-1039 & 2.36\\
  CdSe & mp-1070 & 0.55\\
  GaSb & mp-1156 & 0.36\\
  BP & mp-1479 & 1.46\\
  MoSe$_2$ & mp-1634 & 1.41\\
  BN & mp-1639 & 4.64\\
  YbTe & mp-1779 & 1.52\\
  SnS & mp-1876 & 0.95\\
  SnTe & mp-1883 & 0.66\\
  GeTe & mp-2612 & 0.61\\
  AlSb & mp-2624 & 1.26\\
  CdSe & mp-2691 & 0.50\\
  SnSe & mp-2693 & 0.82\\
  CdSnAs$_2$ & mp-3829 & 0.30\\
  GaCuTe$_2$ & mp-3839 & 0.55\\
  ZnGeAs$_2$ & mp-4008 & 0.56\\
  ZnGeP$_2$ & mp-4524 & 1.20\\
  GaAgTe$_2$ & mp-4899 & 0.19\\
  CdSnP$_2$ & mp-5213 & 0.67\\
  GaCuS$_2$ & mp-5238 & 0.70\\
  SnS & mp-10013 & 0.23\\
  BAs & mp-10044 & 1.25\\
  GeSe & mp-10759 & 0.44\\
  MgSe & mp-10760 & 1.97\\
  CdTe & mp-12779 & 0.61\\
  MgSe & mp-13031 & 2.54\\
  MgTe & mp-13033 & 2.31\\
  TePb & mp-19717 & 1.05\\
  InAs & mp-20305 & 0.30\\
  InP & mp-20351 & 0.46\\
  InAgSe$_2$ & mp-20554 & 0.36\\
  InN & mp-22205 & 0.47\\
  AgI & mp-22894 & 1.39\\
  CuI & mp-22895 & 1.17\\
  CuBr & mp-22913 & 0.48\\
  CuCl & mp-22914 & 0.80\\
  AgI & mp-22919 & 1.00\\
  AgI & mp-22925 & 1.72\\
  Br & mp-23154 & 1.32\\
  TlI & mp-23197 & 2.25\\
  AgBr & mp-23231 & 0.79\\
  BC$_2$N & mp-30148 & 2.10\\
  CuI & mp-569346 & 1.21\\
  Hg & mp-569360 & 0.22\\
  Ga$_2$Os & mp-570875 & 0.66\\
  BC$_2$N & mp-629458 & 1.84\\
  InP & mp-966800 & 0.51\\
  GeC & mp-1002164 & 1.84\\
  TlP & mp-1007776 & 0.12\\
  BC$_2$N & mp-1008523 & 1.64\\
  BP & mp-1008559 & 1.07\\
  OsC & mp-1009540 & 0.17\\
  SiSn & mp-1009813 & 0.41\\
  ZnCdSe$_2$ & mp-1017534 & 1.85\\
  MgSe & mp-1018040 & 2.57\\
  AlSb & mp-1018100 & 0.91\\
  AlBi & mp-1018132 & 0.30\\
  Ge & mp-1067619 & 0.791\\
  Ga$_2$Ru & mp-1072429 & 0.12\\
  ZnCd$_3$Se$_4$ & mp-1078597 & 1.72\\
  BC$_2$N & mp-1079201 & 1.17\\
  Ge & mp-1198022 & 0.67\\
  \hline
\end{longtable}
\end{center}

The constructed dataset consists of compounds formed by all possible combinations of surfaces, interfaces, nanostructures, compositions and structures. We note that this complexity is not neccessarily reflected in the descriptors. Additionally, we acknowledge that many compositions deemed as suitable candidates consists of either rare or dangerous elements. By utilizing an enormously large database as Materials Project, we have to account for their ultimate goal - to model all possible materials and their properties. The automated process of adding an entry to their database does not neccessarily contain all relevant information about a respective material. This is information that needs to be added manually.

Furthermore, we have utilized data obtained from HT-DFT and HT-methods. Indeed, there are possible errors associated with every step, starting from an initial calculation, adding of data in the database, gathering of data, featurization of data, preprocessing of data, data mining and finally training a model and making a prediction. Unfortunately, if an error has happened in the first part of the process, the error follows the entire process and will get increasingly harder to detect. Therefore, we are dependent on that the Materials Project has obtained data with high quality, and we note that it is likely that there are errors present in our data. % the chance of an error present in our data is most likely present.

Motivated by our findings, we believe that further computational, experimental or theoretical verification after a prediction of a possible promising material remains an important step in this work. This step is part of the workflow for novel materials discovery, which is visualized in \autoref{fig:ht-workflow}. Nevertheless, we have provided an exploratory analysis for discovery of novel materials to be used in QT. Considering the amount of materials predicted as suitable candidates by our models and approaches, we hope it encourage further studies and identification of possibe new material hosts.

 %In particular, comment about the quantum effect not present in the dataset, and dft's unability to model.
