\chapter{Material Science Databases}
There are multiple different databases for material science available for every day use, some of them completely open-source while others commercial. This chapter will give a brief overview of databases available for computational material science, and will serve as a toolbox for the speciality of each respective data base.

A quick search online will reveal the tremendous escalation of effort for big-data driven material science the last few years, resulting in several databases. We will here distinguish between a \textit{cloud service}, which is a place to store independent databases for research and commercial purposes, and a \textit{database}, which is an organized collection of structured information. As an example, a cloud service can store several databases, but a database cannot host a cloud service.

To limit the quest of databases, we have restricted the search for databases and cloud services to include inorganic compounds obtained by first-principles calculations. Table \ref{tab:databases} and \ref{tab:cloud_service} shows the databases and cloud services that meets the given criteries, respectively.

\clearpage
%The materials genome initiative

%This piece of research is obtained through open access for institutional and educational purposes,

%Some of the open source databases are Automatic-FLOW for Materials Discovery (AFLOW), Khazana, Materials Discovery (NOMAD), Computational Materials Repository (CMR), NIMSMatNavi, ICSD, Predictive Integrated Structural Materials Science (PRISMS), Materials project, Citrine Informatics, The Materials Platform for Data Science, The Materials Data Fascility, Open Quantum Materials Database (OQMD) and Jarvis. Many of the databases are equipped with their own API, while other requires third-party libraries to extract their information. In addition, for some it is required to obtain an API-key (which) to perform a query,

%This proves that the challenge of finding databases are non-existent, while the challenge of choosing the relevant databases are of a great extent.

\begin{table}[!ht]
\centering
\begin{tabular}{clclclclc}
  \hline
  \hline
  Database & API/REST & Free access & Number of compounds \\
  \hline
  AFLOW & REST & True &  3.27 M\\
  MP \cite{Jain2013}   & MAPI \cite{Ong2015} & True &  0.66 M\\
  OQMD \cite{Kirklin2015, Saal2013} & RESTful API (qmpy, matminer)& True &  0.64 M\\
  ICSD \cite{Levin2020} & RESTful API & False &  0.21 M\\
  Jarvis-DFT  & API & True &  0.04 M\\
  \hline
  \hline
\end{tabular}
\caption{Databases of computational material science sorted after number of compounds. Abbreviations used are Novel Materials Discovery (NOMAD), Automatic-FLOW for Materials Discovery (AFLOW), Materials Project (MP), Inorganic Crystal Structure Database (ICSD) and Open Quantum Materials Database (OQMD).}
\label{tab:databases}
\end{table}

\begin{table}[!ht]
\centering
\noindent\makebox[\textwidth]{
\begin{tabular}{clclcl}
  \hline
  \hline
  Cloud service & API/REST & Free access  \\
  \hline
  NoMaD & API & True  \\
  CMR \cite{Landis2012} & ASE & RESTful API \\
  MatNavi & API & True  \\
  PRISMS & REST & True \\
  Citrine & API & False \\
  MPDS & API & False \\
  MDF & API & False \\
  \hline
  \hline
\end{tabular}
}
\caption{Cloud services that offers database-storage. Abbreviations used are Computational Materials Repository (CMR), NIMS Materials Database (MatNavi), PRedictive Integrated Structural Materials Science (PRISMS), Materials Platform for Data Science (MPDS) and the Materials Data Fascility (MDF).}
\label{tab:cloud_service}
\end{table}

\clearpage

\section{Fundamentals of a database}

% REST

%API

%RESTful HTTP API

\subsection{modules}
Many of the databases share convenient modules that are used to adapt, visualize, calculate or predict properties, making it easier for scientists to utilise the databases.
%matminer

The Atomic Simulation Environment (ASE) is an environment in the Python programming language that includes several tools and modules for setting up, modifying and analyze atomistic simulations \cite{Larsen2017}. It is particularly used together with the cloud service Computational Materials Repository (CMR).

Another commonly used module is the Python Materials Genomics (pymatgen) \cite{Ong2013}. This is a well-documented open module with both introductory and advanced use cases written in Jupyter Notebook for easy reproducibility, and is integrated with the Materials Project REST API.

The Materials Project is also behind a library named matminer \cite{Ward2018}, which is an open-source software platform written in Python. Matminer provides modules to extract data sets from many cloud-services and databases, with exampels in table \ref{tab:databases} and \ref{tab:cloud_service}, it can extract features from images (such as the band gap of a compound), and have modules for visualization of properties.

\section{Databases and cloud services}


\subsection{Novel Materials Discovery}

The Novel Materials Discovery (NOMAD) \cite{Draxl2019} Repository is an open-access platform for sharing and utilizing computational materials science data. NOMAD also consists of several branches such as NOMAD Archieve, which is the representation of the NOMAD repository parsified into a code-independent format, NOMAD Encyclopedia, which is a graphical user interface (GUI) for characterizing materials, and lastly NOMAD Analytics Toolkit, which includes early-development examples of artificial-intelligence tools \cite{Draxl2019}.

Databases that are a part of NOMAD data collection includes Materials Project, the Open Quantum Materials Database and AFLOW. They are all based on the underlying quantum engine Vienna ab initio simulation package (VASP) \cite{Kresse1996}, which is a software based on DFT.

\subsection{Materials project}

Materials project \cite{Jain2013} is an open source project that offers a variety of properties of over one hundred thousand of inorganic crystalline materials. It is known as the initiator of materials genomics and has as its mission to accelerate the discovery of new technological materials, with an emphasis on batteries and electrodes, through advanced scientific computic and innovative design .

It is built upon over 60 features\footnote{All features can be viewed in the documentation of the project: github.com/materialsproject/mapidoc/master/materials}, some features being irrelevant for some materials while fundamental for others. The data is divided into three different branches, where the first can be described as basic properties of materials including over $30$ features, while the second branch describes experimental thermochemical information. The last branch yields information about a particular calculation, in particular information that's relevant for running a DFT script. %Some features, eg. density of state (dos), bandstructure, Pourbaix diagram etc., are characterized by using the module pymatgen.

Every compound has an initial relaxation of cell and lattice parameters performed using a $1000 k$-point mesh to ensure that all properties calculated are representative of the idealized unit cell for each respective crystal structure. The functional GGA is used to calculate band structures, while for transition metals it is applied $+U$ correction to correct for correlation effects in d- and f-orbital systems that are not addressed by GGA calculations \cite{Wang2006}. The thermodynamic stability for each phase with respect to decomposition, is also calculated. This is denoted as E Above Hull, with a value of zero is defined as the most stable phase at a given composition, while larger positive values indicate increased instability.

Each material contains multiple computations for different purposes, resulting in different 'tasks'. The reason behind this is that each computation has a purpose, such as to calculate the band structure or energy. Therefore, it is possible to receive several tasks for one material which results in more features per material.
\subsection{AFLOW}

The AFLOW\cite{Curtarolo2012, Curtarolo2012a, Calderon2015} repository is an automatic software framework for the calculations of a wide range of inorganic material properties. They utilise the GGA-PBE functional within VASP with projector-augmented wavefunction (PAW) potentials to relax twice and optimize the ICSD-sourced structur. They are using a $3000-6000$ $k$-point mesh, indicating a more computationally expensive calculation compared to the Materials Project. Next, the band structure is calculated with an even higher $k$-point density, in addition to the $+U$ correction term for most occupied d- and f-orbital systems, resulting in a standard band gap \cite{Setyawan2010}. Furthermore, they apply a standard fit gathered from a study of DFT-computed versus experimentally measured band gap widths to the initial calculated value, obtaining a fitted band gap \cite{Setyawan2011}.

AFLOW-ML \cite{Isayev2017} is an API that uses machine learning to predict thermomechanical and electronic properties based on the chemical composition and atomic structure alone, which they denote as \textit{fragment descriptors}. They start with applying a classification model to predict if a compound is either a metal or an insulator, where the latter is confirmed with an additional regression model to predict the band gap width. To be able to predict properties on an independent data set, they utilise a fivefold cross validation process for each model. They report a $93$\% prediction success rate of their initial binary classification model, whereas the majority of the wrongful predictions are narrow-gap semiconductors. The authors does not compare their predicted band gap to experimental values, but it is found that $93$\% of the machine-learning-derived values are within $25$\% of the DFT $+U$-calculated band gap width \cite{Ferrenti2020}.


\subsection{Open Quantum Materials Database}

The Open Quantum Materials Database (OQDM) \cite{Saal2013, Kirklin2015} is a free and available database of DFT-calculations \footnote{download the entire database:  http://www.oqmd.org/download/}. It has included thermodynamic and structural properties of more than $600.000$ materials, including all unique entries in the Inorganic Crystal Structure Database (ICSD) consisting of less than $34$ atoms.

The DFT calculations are performed with the VASP software whereas the electron exchange and correlation are described with the GGA-PBE, while using the PAW potentials. They relax a structure using $4000-8000$ $k$-point mesh, indicating an even increasing computational expensive calculation than AFLOW again. Several element-specific settings are included such as using the $+U$ extension for various transition metals, lanthanides and actinides. In addition, any calculation containing 3d or actinide elements are spin-polarized with a ferromagnetic alignment of spins to capture possible magnetism. However, the authors note that this approach does not capture complex magnetic, such as antiferromagnetism, which has been found to result in substantial errors for the formation energy \cite{Stevanovic2012}.

\subsection{JARVIS}

Joint Automated Repository for Various Integrated Simulations (JARVIS) \cite{Choudhary2020} - DFT is an open database based based on the VASP software to perform a variety of material property calculations. It consists of roughly $40.000$ 3D and $1.000$ 2D materials using the vdW-DF-OptB88 van der Waals functional, which was originally designed to improve the approximation of properties of two-dimensional van der Waals materials, but has also shown to be effective for bulk materials \cite{Thonhauser2007, Klimes2011}. The functional has shown accurate predictions for lattice-parameters and energetics for both vDW and non-vdW bonded materials  \cite{Choudhary2018}.

Structures included in the data set are originally taken from the materials project, and then re-optimized using the OPT-functional. Finally, the combination of the OPT and modified Becke-Johnson (mBJ) functionals are used to obtain a representative band gap of each structure, since both have shown unprecedented accuracy in the calculation of band gap compared to any other DFT-based calculation methods \cite{Choudhary2018a}.


The JARVIS-DFT database is part of a bigger platform that includes JARVIS-FF, which is the evaluation of classical forcefield with respect to DFT-data, and JARVIS-ML, which consists of 25 machine learning to predict properties of materials. In addition, JARVIS-DFT also includes a data set of 1D-nanowire and 0D-molecular materials, yet not publically distributed.

\section{}
