\chapter{Material Science Databases}
There are multiple different databases for material science available for every day use, some of them completely open-source while others commercial. In this chapter we will investigate the different databases and what their scope of speciality is.

A quick search online will reveal the tremendous escalation of effort for big-data driven material science the last few years, resulting in several databases. We will here distinguish between a cloud service, which is a place to store independent databases for research and commercial purposes, and a database, which is an organized collection of structured information. As an example, a cloud service can store several databases, but a database cannot host a cloud service.

To limit the quest of databases, we have restricted the search for databases and cloud services to include inorganic compounds obtained by first-principles calculations. Table \ref{tab:databases} and \ref{tab:cloud_service} shows the databases and cloud services that meets the given criteries, respectively.

%The materials genome initiative

%This piece of research is obtained through open access for institutional and educational purposes,

%Some of the open source databases are Automatic-FLOW for Materials Discovery (AFLOW), Khazana, Materials Discovery (NOMAD), Computational Materials Repository (CMR), NIMSMatNavi, ICSD, Predictive Integrated Structural Materials Science (PRISMS), Materials project, Citrine Informatics, The Materials Platform for Data Science, The Materials Data Fascility, Open Quantum Materials Database (OQMD) and Jarvis. Many of the databases are equipped with their own API, while other requires third-party libraries to extract their information. In addition, for some it is required to obtain an API-key (which) to perform a query,

%This proves that the challenge of finding databases are non-existent, while the challenge of choosing the relevant databases are of a great extent.

\clearpage

\begin{table}[!ht]
\centering
\begin{tabular}{clclclclc}
  \hline
  \hline
  Database & API/REST & Free access & Number of compounds \\
  \hline
  AFLOW \cite{Curtarolo2012} & REST & True &  3.27 M\\
  MP \cite{Jain2013}   & MAPI \cite{Ong2015} & True &  0.66 M\\
  OQMD \cite{Kirklin2015, Saal2013} & RESTful API (qmpy, matminer)& True &  0.64 M\\
  %ICSD \cite{Levin2020} & RESTful API & False &  0.21 M\\
  Jarvis-DFT \cite{Choudhary2020} & API & True &  0.04 M\\
  \hline
  \hline
\end{tabular}
\caption{Databases of computational material science sorted after number of compounds. Abbreviations used are Novel Materials Discovery (NOMAD), Automatic-FLOW for Materials Discovery (AFLOW), Materials Project (MP), Inorganic Crystal Structure Database (ICSD) and Open Quantum Materials Database (OQMD).}
\label{tab:databases}
\end{table}

\begin{table}[!ht]
\centering
\noindent\makebox[\textwidth]{
\begin{tabular}{clclcl}
  \hline
  \hline
  Cloud service & API/REST & Free access  \\
  \hline
  NoMaD \cite{Draxl2019}& API & True  \\
  CMR \cite{Landis2012} & ASE & RESTful API \\
  MatNavi & API & True  \\
  PRISMS & REST & True \\
  Citrine & API & False \\
  MPDS & API & False \\
  MDF & API & False \\
  \hline
  \hline
\end{tabular}
}
\caption{Cloud services that offers database-storage. Abbreviations used are Computational Materials Repository (CMR), NIMS Materials Database (MatNavi), PRedictive Integrated Structural Materials Science (PRISMS), Materials Platform for Data Science (MPDS) and the Materials Data Fascility (MDF).}
\label{tab:cloud_service}
\end{table}

\clearpage

\section{Fundamentals of a database}

% REST

%API

%RESTful HTTP API

\subsection{modules}
Many of the databases share a common and convenient factor that happens to easy the life of their users. The modules used to adapt, visualize, calculate or predict have become established in their field of speciality.

%matminer

The Atomic Simulation Environment (ASE) is an environment in the Python programming language that includes several tools and modules for setting up, modifying and analyze atomistic simulations \cite{Larsen2017}. It is particularly used together with the cloud service Computational Materials Repository (CMR).

Another commonly used module is the Python Materials Genomics (pymatgen) \cite{Ong2013}. This is a well-documented open module with both introductory and advanced use cases written in Jupyter Notebook for easy reproducibility and offline-scenarios. It is integrated with the Materials Project REST API (MAPI) by using a wrapper.

\section{Novel Materials Discovery}

The Novel Materials Discovery (NOMAD) Repository is an open-access platform for sharing and utilizing computational materials science data. NOMAD also consists of several branches such as NOMAD Archieve, which is the representation of the NOMAD repository parsified into a code-independent format, NOMAD Encyclopedia, which is a graphical user interface (GUI) for characterizing materials, and lastly NOMAD Analytics Toolkit, which includes early-development examples of artificial-intelligence tools \cite{Draxl2019}.

Databases that are a part of NOMAD data collection includes Materials Project, the Open Quantum Materials Database and AFLOW which are all based on the underlying quantum engine VASP \footnote{https://www.vasp.at}.

\subsection{Materials project}

Materials project is an open source project that offers a variety of properties of over one hundred thousand of inorganic crystalline materials. It is known as the initiator of materials genomics and has as its mission to accelerate the discovery of new technological materials, with an emphasis on batteries and electrodes, through advanced scientific computic and innovative design\footnote{add link MP here}.

It is built upon over 60 features\footnote{All features can be viewed in the documentation of the project: github.com/materialsproject/mapidoc/master/materials}, some features being irrelevant for some materials while fundamental for others. Almost all of the data in the project is calculated using a theoretical technique called Density Functional Theory. The data is divided into three different branches, where the first can be described as basic properties of materials including over $30$ features, while the second branch describes experimental thermochemical information. The last branch yields information about a particular calculation, in particular information that's relevant for running a DFT script. Some features, eg. density of state (dos), bandstructure, Pourbaix diagram etc., are characterized by using the module pymatgen.

Each material contains mutiple computations for different purposes, resulting in different 'tasks'. The reason behind this is that each computation has a purpose, eg. to calculate band gap, energy etc. Therefore, it is possible to receive several tasks for one material which results in more features per material.

\subsection{Open Quantum Materials Database}

The Open Quantum Materials Database (OQDM) is a completely free and available database of DFT-calculations \footnote{download the entire database:  http://www.oqmd.org/download/}. It has included thermodynamic and structural properties of more than $600.000$ materials, including all unique entries in the Inorganic Crystal Structure Database (ICSD) consisting of less than $34$ atoms \cite{Kirklin2015}.

For general DFT-settings, see http://oqmd.org/documentation/vasp. (read a bit more about VASP before entering this labyrinth)

%\subsection{AFLOW}


\subsection{Joint Automated Repository for Various Integrated Simulations}

Joint Automated Repository for Various Integrated Simulations (JARVIS) - DFT is an open database based solely on DFT-calculations. It consists of roughly $40.000$ 3D and $1.000$ 2D materials using the vdW-DF-OptB88 van der Waals functional. This functional has shown accurate predictions for lattice-parameters and energetics for both vDW and non-vdW bonded materials  \cite{Choudhary2018}. There does also exist more data including 1D-nanowire and 0D-molecular materials, yet not publically distributed.

The JARVIS-DFT database is part of a bigger platform that includes JARVIS-FF, which is the evaluation of classical forcefield with respect to DFT-data, and JARVIS-ML, which consists of 25 machine learning to predict properties of materials.

\section{}
