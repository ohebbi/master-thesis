\chapter{Material Science Databases}

There are multiple different databases for material science discovery available for every day use, some of them completely open-source while others are commercial. This chapter will give a brief overview of databases available for computational material science, and will serve as a toolbox for how to request information and what kind of python packages exist to process that information.

%The materials genome initiative


%Some of the open source databases are Automatic-FLOW for Materials Discovery (AFLOW), Khazana, Materials Discovery (NOMAD), Computational Materials Repository (CMR), NIMSMatNavi, ICSD, Predictive Integrated Structural Materials Science (PRISMS), Materials project, Citrine Informatics, The Materials Platform for Data Science, The Materials Data Fascility, Open Quantum Materials Database (OQMD) and Jarvis. Many of the databases are equipped with their own API, while other requires third-party libraries to extract their information. In addition, for some it is required to obtain an API-key (which) to perform a query,

%This proves that the challenge of finding databases are non-existent, while the challenge of choosing the relevant databases are of a great extent.

\section{Fundamentals of a database}

A quick search online will reveal the tremendous escalation of effort for big-data driven material science the last few years, resulting in several databases that stores ab-initio calculation details and results. We will here distinguish between a \textit{cloud service}, which is a place to store independent databases for research and commercial purposes, and a \textit{database}, which is an organized collection of structured information. As an example, a cloud service can store several databases, but a database cannot host a cloud service.

To limit the quest of databases, we have restricted the search for databases and cloud services to include inorganic compounds obtained experimentally or by first-principles calculations, in particular DFT-calculations using \textit{Vienna ab initio simulation package} (VASP) \cite{Kresse1996}. VASP is a software for atomic scale materials programming. Table \ref{tab:databases} and \ref{tab:cloud_service} shows a selection of databases and cloud services that meets the given criteries, respectively.

\subsection{API and HTTP requests}

To extract information from a database it is convenient to interact through an \textit{API} (Application Programming Interface), which defines important variables such as the kind of requests to be made, how to make them and the data format for transmission. Importantly, this permits communication between different software medias. An API is entirely customizable, and can be made to extend existing functionality or tailormade for specific user-demanding modules.

The APIs that will be encountered is handled by the use of \textit{HTTP} (Hypertext Transfer Protocol), which in its simplest form is a protocol that allows the fetching of resources. The protocol is client-server based, such as the client is requesting information and the server is responding to the request.

The most common HTTP-methods are GET, POST and HEAD, which are used to either retrieve, send, or get information about data, respectively. The latter request is usually done before a GET-method for requests considering large amount of data, since this can be a significant variable for the client's bandwith and load time. Following a request, the server normally responds with one of the status codes in table \ref{tab:requests}.

\begin{table}[!ht]
\centering
\noindent\makebox[\textwidth]{
\begin{tabular}{M{2.5cm} M{11.0cm}}
  \hline
  \hline
   Status code & Description \\
  \hline
  2xx & OK - request was successful \\
  3xx & Resource was redirected \\
  4xx & Request failed due to either unsuccessful authentication or client error. \\
  5xx & Request failed due to server error. \\
  \hline
  \hline
\end{tabular}
}
\caption{Numeric status code for response. The leftmost digit decide the type of response, while the two follow-up digits depends on the implemented API.}
\label{tab:requests}
\end{table}

A RESTful (Representational State Transfer) allows users to communicate with a server via a HTTP using a REST Architectural Style \cite{Battle2008}. This enables the utilisation of Uniform Resource Identifiers (URI), where each object is represented as a unique resource and can be requested in a uniform manner. Importantly, this allows the use of both URIs and HTTP methods in an API, such that an object is represented by an unique URI whereas a HTTP-method can act on the object. This action will then return either the result of the action, or structured data that represents the object.

\begin{table}[!ht]
\centering
%\begin{tabular}{clclclclc}
\noindent\makebox[\textwidth]{
  \begin{tabular}{M{3.0cm} M{3.5cm} M{2.5cm} M{2.5cm}}
  \hline
  \hline
  Database & API & Free educational access & Number of entries \\
  \hline
  AFLOW & REST & True &  3.27 M\\
  OQMD \cite{Kirklin2015, Saal2013} & RESTful API (qmpy, matminer)& True &  0.82 M\\
  MP \cite{Jain2013}   & MAPI \cite{Ong2015} & True &  0.71 M\\
  ICSD \cite{Levin2020} & RESTful API & False &  0.21 M\\
  Jarvis-DFT  & API & True &  0.04 M\\
  \hline
  \hline
\end{tabular}
}
\caption{Databases of computational material science sorted after number of compounds. Abbreviations used are Novel Materials Discovery (NOMAD), Automatic-FLOW for Materials Discovery (AFLOW), Materials Project (MP), Inorganic Crystal Structure Database (ICSD) and Open Quantum Materials Database (OQMD). The number of entries can give the wrong perception of size of each respective database, as it does not visualise how many calculations have been done for each entry, nor if there might be duplicates.}
\label{tab:databases}
\end{table}

\begin{table}[!ht]
\centering
\noindent\makebox[\textwidth]{
  \begin{tabular}{M{3.0cm} M{3.0cm} M{3.0cm}}
  \hline
  \hline
  Cloud service & API/REST & Free educational access  \\
  \hline
  NoMaD & API & True  \\
  CMR \cite{Landis2012} & ASE & RESTful API \\
  MatNavi & API & True  \\
  PRISMS & REST & True \\
  Citrine & API & True \\
  MPDS & API & False \\
  MDF & API & False \\
  \hline
  \hline
\end{tabular}
}
\caption{Cloud services that offers database-storage. Abbreviations used are Computational Materials Repository (CMR), NIMS Materials Database (MatNavi), PRedictive Integrated Structural Materials Science (PRISMS), Materials Platform for Data Science (MPDS) and the Materials Data Fascility (MDF).}
\label{tab:cloud_service}
\end{table}

\subsection{Open-source Python libraries for material analysis}
Many of the databases share convenient modules that are used to adapt, visualize, calculate or predict properties, making it easier for scientists to utilise the databases.
%matminer
%seamless module compatibility
The Atomic Simulation Environment (ASE) is an environment in the Python programming language that includes several tools and modules for setting up, modifying and analyze atomistic simulations \cite{Larsen2017}. It is in particular used together with the cloud service Computational Materials Repository (CMR).

Another commonly used module is the Python Materials Genomics (pymatgen) \cite{Ong2013}. This is a well-documented open module with both introductory and advanced use cases written in Jupyter Notebook for easy reproducibility, and is integrated with the Materials Project RESTful API.

The Materials Project is also behind a library named matminer \cite{Ward2018}, which is an open-source software platform written in Python. Matminer provides modules to extract data sets from many cloud-services and databases, with exampels in table \ref{tab:databases} and \ref{tab:cloud_service}, it can extract features from images (such as the band gap of a compound), and have modules for visualization of properties.

\section{Databases and cloud services}

Every database has its own speciality, and no two databases are the same. There exists entries that are fundamentally identical in several databases, but with different properties as a consequence of parameters used, such as the functional utilised in VASP or the relaxation scheme. This section digs up what exactly is each respective database's claim to fame.

\subsection{Novel Materials Discovery}

The Novel Materials Discovery (NOMAD) \cite{Draxl2019} Repository is an open-access platform for sharing and utilizing computational materials science data. NOMAD also consists of several branches such as NOMAD Archieve, which is the representation of the NOMAD repository parsified into a code-independent format, NOMAD Encyclopedia, which is a graphical user interface (GUI) for characterizing materials, and lastly NOMAD Analytics Toolkit, which includes early-development examples of artificial-intelligence tools \cite{Draxl2019}.

Databases that are a part of NOMAD data collection includes Materials Project, the Open Quantum Materials Database and AFLOW. They are all based on the underlying quantum engine VASP.

\subsection{Materials project}

Materials project \cite{Jain2013} is an open source project that offers a variety of properties of over one hundred thousand of inorganic crystalline materials. It is known as the initiator of materials genomics and has as its mission to accelerate the discovery of new technological materials, with an emphasis on batteries and electrodes, through advanced scientific computic and innovative design .

 %Some features, eg. density of state (dos), bandstructure, Pourbaix diagram etc., are characterized by using the module pymatgen.

Every compound has an initial relaxation of cell and lattice parameters performed using a $1000 k$-point mesh to ensure that all properties calculated are representative of the idealized unit cell for each respective crystal structure. The functional GGA is used to calculate band structures, while for transition metals it is applied $+U$ correction to correct for correlation effects in d- and f-orbital systems that are not addressed by GGA calculations \cite{Wang2006}. The thermodynamic stability for each phase with respect to decomposition, is also calculated. This is denoted as E Above Hull, with a value of zero is defined as the most stable phase at a given composition, while larger positive values indicate increased instability.

Each material contains multiple computations for different purposes, resulting in different 'tasks'. The reason behind this is that each computation has a purpose, such as to calculate the band structure or energy. Therefore, it is possible to receive several tasks for one material which results in more features per material.

\subsection{AFLOW}

The AFLOW\cite{Curtarolo2012, Curtarolo2012a, Calderon2015} repository is an automatic software framework for the calculations of a wide range of inorganic material properties. They utilise the GGA-PBE functional within VASP with projector-augmented wavefunction (PAW) potentials to relax twice and optimize the ICSD-sourced structur. They are using a $3000-6000$ $k$-point mesh, indicating a more computationally expensive calculation compared to the Materials Project. Next, the band structure is calculated with an even higher $k$-point density, in addition to the $+U$ correction term for most occupied d- and f-orbital systems, resulting in a standard band gap \cite{Setyawan2010}. Furthermore, they apply a standard fit gathered from a study of DFT-computed versus experimentally measured band gap widths to the initial calculated value, obtaining a fitted band gap \cite{Setyawan2011}.

AFLOW-ML \cite{Isayev2017} is an API that uses machine learning to predict thermomechanical and electronic properties based on the chemical composition and atomic structure alone, which they denote as \textit{fragment descriptors}. They start with applying a classification model to predict if a compound is either a metal or an insulator, where the latter is confirmed with an additional regression model to predict the band gap width. To be able to predict properties on an independent data set, they utilise a fivefold cross validation process for each model. They report a $93$\% prediction success rate of their initial binary classification model, whereas the majority of the wrongful predictions are narrow-gap semiconductors. The authors does not compare their predicted band gap to experimental values, but it is found that $93$\% of the machine-learning-derived values are within $25$\% of the DFT $+U$-calculated band gap width \cite{Ferrenti2020}.


\subsection{Open Quantum Materials Database}

The Open Quantum Materials Database (OQDM) \cite{Saal2013, Kirklin2015} is a free and available database of DFT-calculations. It has included thermodynamic and structural properties of more than $600.000$ materials, including all unique entries in the Inorganic Crystal Structure Database (ICSD) consisting of less than $34$ atoms.

The DFT calculations are performed with the VASP software whereas the electron exchange and correlation are described with the GGA-PBE, while using the PAW potentials. They relax a structure using $4000-8000$ $k$-point mesh, indicating an even increasing computational expensive calculation than AFLOW again. Several element-specific settings are included such as using the $+U$ extension for various transition metals, lanthanides and actinides. In addition, any calculation containing 3d or actinide elements are spin-polarized with a ferromagnetic alignment of spins to capture possible magnetism. However, the authors note that this approach does not capture complex magnetic, such as antiferromagnetism, which has been found to result in substantial errors for the formation energy \cite{Stevanovic2012}.

\subsection{JARVIS}

Joint Automated Repository for Various Integrated Simulations (JARVIS) \cite{Choudhary2020} - DFT is an open database based based on the VASP software to perform a variety of material property calculations. It consists of roughly $40.000$ 3D and $1.000$ 2D materials using the vdW-DF-OptB88 van der Waals functional, which was originally designed to improve the approximation of properties of two-dimensional van der Waals materials, but has also shown to be effective for bulk materials \cite{Thonhauser2007, Klimes2011}. The functional has shown accurate predictions for lattice-parameters and energetics for both vDW and non-vdW bonded materials  \cite{Choudhary2018}.

Structures included in the data set are originally taken from the materials project, and then re-optimized using the OPT-functional. Finally, the combination of the OPT and modified Becke-Johnson (mBJ) functionals are used to obtain a representative band gap of each structure, since both have shown unprecedented accuracy in the calculation of band gap compared to any other DFT-based calculation methods \cite{Choudhary2018a}.


The JARVIS-DFT database is part of a bigger platform that includes JARVIS-FF, which is the evaluation of classical forcefield with respect to DFT-data, and JARVIS-ML, which consists of 25 machine learning to predict properties of materials. In addition, JARVIS-DFT also includes a data set of 1D-nanowire and 0D-molecular materials, yet not publically distributed.

\section{Practical data extraction with Python-examples}

For this section, we will show practical examples of how to extract data that fulfill the criteria for a material to host a qubit candidate given in the theory part. We will begin with the database of Materials Project, and then restrict the query thereafter. This data mining process is reproducable as a jupyter notebook\footnote{add and insert DOI for JN data-mining.ipynb} and the databases in question are the ones refered to in the previous section.

Instead of setting up a multiple HTTP-methods, we will here take a look at the easiest method at obtaining data from each database. This includes looking into the APIs that supports data-extraction and that are recommended by each respective database.

The range of data in a database can consist of data from a few entries up to an unlimited amount of entries with even further optional parameters, and has limitless use in applications. However, the amount of data in a database is irrelevant if the data is inaccessible. Therefore, we present an elementary formula which we will use in evaluating if a database is accessible or not. It is defined as
\begin{align}
  \text{Accessibility} = \frac{\text{Extraction speed}}{\text{Amount of data}}.
\end{align}
A large accessibility term implies an ease in extracting information. This formula does not depend on how accessible a database' user interface is, but a discussion of documentation and user interface will be included in the examples.

\subsection{Materials Project }

The most up-to-date version of Materials Project can be extracted using the python package pymatgen, which is integrated with Materials Project REST API. Other retrievel tools that is dependent on pymatgen includes matminer, with the added functionality of returning a pandas dataframe. Copies of Materials Project are added frequently to cloud services such as Citrine Informatics, but the latest added entries to Materials Project cannot be guaranteed in such a query.

Entries in Materials Project are characterized using more than 60 features\footnote{All features can be viewed in the documentation of the project: https://github.com/materialsproject/mapidoc/master/materials}, some features being irrelevant for some materials while fundamental for others. The data is divided into three different branches, where the first can be described as basic properties of materials including over $30$ features, while the second branch describes experimental thermochemical information. The last branch yields information about a particular calculation, in particular information that's relevant for running a DFT script.

To extract information from the database, we will be utilising the module pymatgen. This query supports MongoDB query and projection operators\footnote{https://docs.mongodb.com/manual/reference/operator/query/}, resulting in an almost instant query. Thus, Materials Project is regarded as a highly accessible database.

\begin{enumerate}
  \item Register for an account\footnote{https://materialsproject.org}, and generate a secret API-key.
  \item Install pymatgen in the current environment
  \item Import pymatgen.
  \item Set the required critera.
  \item Set the wanted properties.
  \item Apply the query.
\end{enumerate}

The code nippet in code listing \ref{lst:MPQuery} resembles steps $3-6$, with the additional usage of the library Pandas. For this particular query we have set the filter as four items. Firstly, we would like to exclude all spin zero isotopes using the MongoDB operator that matches non of the values specified in the array. Thereafter, we would like to have a compound that is deemed similar to an ICSD entry. All of the resulting entries have be deemed non-magnetic (NM), and lastly, all compounds with polar space groups will be excluded.

\lstinputlisting[language=Python, caption={Practical example of extracting information from Materials Project using pymatgen, resulting in a Pandas DataFrame named entries that contains the properties given after performing a filter on the database. The criteria is given as a JSON, and supports MongoDB operators.}, label={lst:MPQuery}]{theory/code-listings/MPQuery.tex}

\subsection{Citrine Informatics}

Citrine Informatics is a cloud service, which means that the spectrum of stored information varies broadly. We will access research through open access for institutional and educational purposes. Information in Citrine can be stored using a scheme that is broken down into two sections, with private properties for each entry in addition to common fields that are the same for all entries. However, the query happens swiftly and is noted as highly accessible.

In this example, we will gather experimental data using the module matminer. The following steps are required to extract information from Citrine Informatics.

\begin{enumerate}
  \item Register for an account\footnote{https://citrination.com}, and generate a secret API-key.
  \item Install matminer in the current environment.
  \item Import matminer.
  \item Set the required critera.
  \item Set the wanted properties and common fields.
  \item Apply the query.
\end{enumerate}

The code listed in code listing \ref{lst:CIQuery} gives an easy example to steps $2-4$ with experimental data as filter. The resulting query will be returned as a Pandas DataFrame, but it is not neccessary to include the pandas since it is already implemented in the module matminer.

\lstinputlisting[language=Python, caption={Practical example of extracting information from Citrine Informatics using matminer, resulting in a Pandas DataFrame named experimental\_entries that contains the properties given after performing a filter on the database. The criteria is given as a JSON.}, label={lst:CIQuery}]{theory/code-listings/CIQuery.tex}

\subsection{AFLOW}

The query from AFLOW API \cite{Curtarolo2012} supports lazy formatting, which means that the query is just a search and does not return values but rather an object. This object is then used in the query when asking for values. For every object it is neccessary to request the desired property, consequently making the query process significantly more time-demanding than similar queries using APIs such as pymatgen or matminer for Citrine Informatics. Hence, the accessibility is strictly limited to either searching for single compounds or if the user possess sufficient time.

Matminer's data retrievel tool for AFLOW is currently an ongoing issue \cite{Rosenbrock2017}, thus we present in code listing \ref{lst:AFLOWQuery} a function that extracts information from AFLOW and returns a Pandas DataFrame. In contrast to Materials Project and Citrine Informatics, AFLOW does not require an API-key for a query, which reduces the amount of steps to obtain data.

\lstinputlisting[language=Python, caption={Practical example of extracting information from AFLOW. The function can extract all information in AFLOW for a given list of compounds, however, it is a slow method and requires consistent internet connection.}, label={lst:AFLOWQuery}]{theory/code-listings/AFLOWQuery.tex}

\subsection{AFLOW-ML}

In this part, we will be using a machine learning algorithm named AFLOW-ML Property Labeled Material Fragments (PLMF) \cite{Isayev2017} to predict the band gap of structures. This algorithm is compatible with a POSCAR of a compound, which can be generated by the CIF (Crystallographic Information File) that describes a crystal's generic structure. It is possible to download a structure as a poscar by using Materials Project front-end API, but is a cumbersome process to do so individually if the task includes many structures. Extracting the feature of POSCAR is yet to be implemented in the RESful API of pymatgen, thus we demonstrate the versatility of pymatgen with a workaround.

We begin with extracting the desired compounds formula, its material\_id for identification, and their respectful structure in CIF-format from Materials Project. In an iterative process, each CIF-structure is parsed to a pymatgen structure, where pymatgen can read and convert the structure to a POSCAR stored as a Python dictionary. Finally, we can use the POSCAR as input to AFLOW-ML, which will return the predicted band gap of the structure. This iterative process parsing and converting, but is an undemanding process. The function that handles this is presented in code listing \ref{lst:AFLOWMLQuery}.

A significant portion of the process is tied up to obtaining the input-file for AFLOW-ML, and fewer structures will result in an easier process. Nevertheless, we present the following steps in order to receive data from AFLOW-ML.

\begin{enumerate}
  \item Download AFLOWmlAPI\footnote{http://aflow.org/src/aflow-ml/ to the same directory as code listing \ref{lst:AFLOWMLQuery}}.
  \item Getting POSCAR from MP.
  \begin{enumerate}
    \item Apply the query from Materials Project with "CIF", "material\_id" and "full\_formula" as properties.
    \item Insert resulting DataFrame into function defined in code listing \ref{lst:AFLOWMLQuery}.
  \end{enumerate}
    \item Insert POSCAR to AFLOW-ML.
\end{enumerate}

\lstinputlisting[language=Python, caption={Practical example of extracting information from AFLOW-ML. The function will convert a CIF-file (from e.g. Materials Project) to a POSCAR, and will use it as input to AFLOW-ML. In return, one will get the structure's predicted band gap. It should be noted that this requires the AFLOW-ML library in the same directory.}, label={lst:AFLOWMLQuery}]{theory/code-listings/AFLOWMLQuery.tex}

\subsection{JARVIS-DFT}

The newest version of the JARVIS-DFT dataset can be obtained by requesting an account at the official webpage, but with the drawback that an administrator has to either accept or deny the request. Thus, the accessibility of the database is dependent on if there is an active administrator paying attention to the requests. Another approach is to download the database through matminer, however with the limitation of not neccessarily having the latest version of the database. The following steps describes the process of extracting JARVIS-DFT using matminer's convenience loader module, and can be regarded as easily accessible with few lines of code and instantanous download.

\begin{enumerate}
  \item Install matminer in the current environment.
  \item Import matminer.
  \item Load the dataset using code listing \ref{lst:JARVISDFTQuery}.
\end{enumerate}
\lstinputlisting[language=Python, caption={Practical example of extracting information from JARVIS-DFT. For this example, we exclude all metals by removing all non-measured band gaps.}, label={lst:JARVISDFTQuery}]{theory/code-listings/JARVISDFTQuery.tex}

We can observe that there is no advanced search filter when loading the database from matminer. The author of matminer regards this as the user's task, which is easily done through the use of the python library Pandas.
