\chapter{Quantum mechanics}

In the purpose of fully understanding the underlaying physics behind computational material science, we will need to investigate how we can calculate the forces happening inside a crystal. This is a field of expertise where classical models gives inaccurate estimates, thus it is inevitable to start off with some fundamental quantum mechanics.

We will in this thesis only formulate the neccessary theory behind density functional theory, leaving most of the quantum-mechanical world untouched. However, the fundamental theory remains the same and we will start our venture with the single-electron Schrödinger equation.

%canonical variables
%dynamical variables
%operator
%canonical substitusion
\section{The single-electron Schrödringer equation}

As every other introduction to quantum mechanics books, we will start of investigating the Schrödinger equation with only one electron \cite{Griffiths2017}
\begin{align}
    i\hslash \frac{\partial \Psi}{\partial t} = -\frac{\hslash^2}{2m}\nabla^2 \Psi + V\Psi
    \label{eq:Schrödinger}
\end{align}
for a convenient external potential $V_{ext}(r)$ that is independent of time. We will try to look for solutions for (\ref{eq:Schrödinger}) by separating the wave function into a space-dependent and time-dependent function
%Most wavefunctions are solutions to (\ref{eq:Schrödinger}), but if the wavefunction describes a stationary state, the wavefunction has to be an eigenfunction to H for reasons that will become clear shortly.
\begin{align}
  \Psi(r,t) = \psi(r)\phi(t).
  \label{eq:separation}
\end{align}
By inserting ordinary derivatives and dividing each side with equation (\ref{eq:separation}), our Schrödinger equation (\ref{eq:Schrödinger}) now reads
\begin{align}
  i\hslash \frac{1}{\phi(t)}\frac{d\phi(t)}{dt} = - \frac{\hslash^2}{2m} \frac{1}{\psi(r)}\nabla^2 \psi + V(r)
\end{align}

Since the potential function $V(r)$ is independent of time, we observe the time and space dependencies of each side and state the fact that both sides has to be constant. Thus, two intriguing equations unveil themselves;
%captivating?
\begin{align}
  i\hslash \frac{1}{\phi(t)}\frac{d\phi(t)}{dt} = E\phi(t)
  \label{eq:time}
\end{align}
and
\begin{align}
  \frac{\hslash^2}{2m} \frac{1}{\psi(r)}\nabla^2 \psi + V(r) = E\psi(r)
  \label{eq:tise}
\end{align}
where the first equation (\ref{eq:time}) has a general solution $\phi(t) = C \exp (-iEt/\hslash)$ and $C=1$ after normalization, and the second equation (\ref{eq:tise}) is known as time-independent Schrödinger equation. These two equations are connected through the variable $\varepsilon$.

By utilizing variable separation to get equation (\ref{eq:separation}), we find that the wavefunction is describing a stationary state with probability density
\begin{align*}
  \lvert \Psi (r,t)\rvert ^2 &= \Psi^*\Psi \\
  &= \Psi^* e^{iEt/\hslash} \Psi e^{-i Et/\hslash} \\
  &= \lvert \Psi (r)\rvert ^2
\end{align*}
that is independent of time. Conveniently, this is also true for every expectation value; they are all constant in time. We can also try to express this in classical terms regarding the Hamiltonian, which in this scenario is defined as
\begin{align}
    \hat{H}(r, p) = \frac{p^2}{2m} + V(r) = -\frac{\hslash^2}{2m}\nabla^2 + V(r)
\end{align}
simplifying equation \ref{eq:tise} to
\begin{align}
  \hat{H} \psi = E\psi
  \label{eq:tise_nesten}
\end{align}
and we can find the expectation value of the total energy as
\begin{align*}
    \langle H \rangle &= \int \psi^* \hat{H} \psi dr \\
                &= E\int \lvert \Psi \rvert ^2 dr \\
                &= E
\end{align*}
using the fact that expectation values are constant in time for stationary states. Similarly, we can try to estimate the variance of the Hamiltonian,
\begin{align*}
  \sigma_H^2 &= \langle H^2 \rangle - \langle H \rangle ^2 \\
            &= E^2 - E^2 \\
            &= 0
\end{align*}
which appropiately describes that every measurement of the total energy is certain to return the value E.



\section{Eigenfunctions}
So far, we have not given an explanation of what a wavefunction is. As a matter of fact, we have actually found an eigenfunction
\begin{align*}
  \psi_\kappa^*(r,t) = \psi_\kappa e^{-i\varepsilon_\kappa t/\hslash}
\end{align*}
where $\kappa$ denotes the $k$-th eigenfunction and $\varepsilon_\kappa$ is its corresponding energy eigenvalue. The eigenfunctions have distinct energies and have the attribute that they are orthogonal and normalized with respect to
\begin{align*}
  \bra{\psi_\kappa (r,t)} \ket{\psi_{\kappa`} (r,t)} = \delta_{\kappa \kappa'}.
\end{align*}
The state with the lowest energy is called the ground state, and is where it is most likely to find an electron in a single-electron system with no external potential applied.

A general wavefunction can be generated by a summation of eigenfunctions (such as the eigenfunction in the latter case)
\begin{align}
\Psi(r,t) = \sum_\kappa c_\kappa \psi_{\kappa}(r,t),
\end{align}
where $c_\kappa$ is a constant. A general wavefunction does not neccessarily describe stationary states, and consequently does not have distinct energies but is rather represented statistically from the expectation value
\begin{align*}
  E = \sum_{\kappa} \lvert c_\kappa \rvert \varepsilon_\kappa.
\end{align*}
Solving Schrödinger equation for a general wavefunction is rather troublesome. Fortunately, we can use the eigenfunctions instead, transforming equation \ref{eq:tise_nesten} into time-independent Schrödinger equation for eigenfunctions
\begin{align}
  \hat{H} \psi_{\kappa}(r) = \varepsilon_\kappa \psi_\kappa(r).
\end{align}

The shape of en eigenfunction has normally high spatial symmetri that depends on the symmetri of the potential $V_{ext}(r)$ and the boundary conditions \cite{Persson2020}. The study of how atoms in a crystalline interact with each other is of upmost importance when trying to explain macroscopic consequences.

\section{Hartree-Fock approximations}

As we venture along from a one-electron system to a two-electron systen, we encounter a new wavefunction and Hamiltonian that needs to describe two particles, making the two-electron Schrödinger equation read

\begin{align}
  \Big( -\frac{\hslash^2 \nabla_1^2}{2m_e} - \frac{\hslash^2\nabla_2^2}{2m_e}+ \frac{q^2}{\lvert r_1-r_2  \rvert} + V_{ext}(r) \Big) \Psi_\kappa (r_1, r_2) = E_{\kappa} \Psi_\kappa (r_1, r_2),
\end{align}
where the two first terms are the kinetic energies of the electrons, while the third term is a potential that describes the repulsive Coloumb interaction between the two electrons. The last term is the external potential, well known from the earlier scenario with only one electron.

The Hartre approximation to the two-electron wavefunction is to make an \textit{ansatz}, a clever guess, for the wavefunction
\begin{align}
  \Psi(r_1,r_2) = A \cdot \psi_1(r_1) \psi_2(r_2).
\end{align}
The downside with this approach is that the particles are distinguishable and do not obey the Pauli exclusion principle for fermions.

The Hartree-fock approach, however, overcame this challenge and presented an anti-symmetric wavefunction that made the electrons indistinguishable;
\begin{align}
  \Psi(r_1,r_2) = \frac{1}{\sqrt{2}}\Big( \psi_1(r_1) \psi_2  + {\psi_1(r_2)\psi_2(r_1)}\Big)
\end{align}



\clearpage
