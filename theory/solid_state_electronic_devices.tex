\chapter{Identifying materials}
\section{Crystallography}

Solid materials are formed by densely packed atoms. These atoms can randomly occur through the material without any long-range order, which would categorize the material as an \textit{amorphous solid}. Amorphous solids are frequently used in gels, glass and polymers \footnote{Need source}.

However, the atoms can also be found periodic in small regions of the material, classifying the material as a \textit{polycrystalline solid}. All ceramics are polycrystalline with a broad specter of usage ranging from kitchen-porcelain to orthopedical bio-implant \cite{Renganathan2018}.

A third option is to have these atoms arranged infinite periodically, making the material a \textit{crystalline solid} or more commonly named a \textit{crystal}.

The periodicity in a crystal is defined in terms of a symmetric array of points in space called the \textit{lattice}, which can be simplified as either a one-dimensional array, a two-dimensional matrix or a three dimensional vector space depending on the material. At each lattice point we can add an atom to make an arrangement called a \textit{basis}. The basis can be one atom or a cluster of atoms having the same spatial arrangement, making a \textit{crystal}. For every crystal, there exists periodically repeated building blocks called \textit{cells} which represents the entire crystal. The smallest cell possible is called a \textit{primitive cell}, but such a cell only allows lattice points at its corners and it is often quite rigid to work with when the structure becomes complex. As a solution, we will consider the \textit{unit cell}, which allows lattice points on face centers and body centers.


\section{Semiconductors}

Define semiconductors

A material that conducts electrical current is, by definition, a metal. On the other hand, a material that does not conduct electrical current is an insulator. A semiconductor is an element or a compound that is

Semiconductors are elements or compounds that
