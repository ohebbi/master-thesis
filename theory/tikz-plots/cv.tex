\begin{figure}[!ht]
  \centering
\begin{tikzpicture}[node distance=0mm,minimum height=1cm,outer sep=3mm,scale=0.5,>=Latex,font=\footnotesize,
  indication/.style={minimum height=0cm,outer sep=0mm},
  oneblock/.style={transform shape,minimum width=1cm,draw},
  fullset/.style={transform shape,minimum width=10cm,draw}]
    % left part of picture
    \node[fullset,anchor=west] at (0,0) (A) {};
    \node[above=of A.north,indication] (ATXT) {TRAINING SET};
    \node[oneblock,minimum width=2cm,anchor=west,right=of A,fill=lightgray,outer sep=0mm] (A1) {};
    \path (ATXT) -| (A1) node[midway] {TEST SET};
    \node[fullset,anchor=west] at (0,-4) (B) {};
    \foreach \x in {0,1,...,9}
    {
        \draw (B.west) +(\x,0) node[oneblock,anchor=west,draw] {};
    }
    \draw[->] (A) -- (B) node[midway,fill=white,indication] {divide into 10 folds of equal size};

    % right part of picture
    \begin{scope}[xshift=15cm,scale=0.5,local bounding box=rightside box]
    \foreach \x in {0,1}
    {
        \foreach \y in {0,1,...,4}
        {
            \draw (\x*11,0) +(0,-\y*2) node[fullset,anchor=west] {};
            \draw (\x*11,0) +(\x*5+\y,-\y*2) node[oneblock,draw,anchor=west,fill=lightgray] {};
        }
    }
    \coordinate (R) at (rightside box.west);
    \end{scope}

    % connecting arrow
    \draw[->] (B.east) -- +(2.5,0) node[below,align=center,indication] {run experiments\\using 10 different\\partitionings} |- (R);
  \end{tikzpicture}
  \caption{A schematic representation of a $10$-fold cross-validation scheme.}
  \label{fig:cv}
\end{figure}
